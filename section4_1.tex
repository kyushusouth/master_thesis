\subsection{音声合成法}
\subsubsection{全体像}
本実験で用いる学習データセット$\datasetTrain$を
\begin{equation}
    \datasetTrain = \lrc{\lr{\video, \spWaveformGt, \spkEmb, \melGt, \hubertIntGt, \hubertDiscGt}}_{\numLower = 1}^{\numUpper}
\end{equation}
とする.各$\numLower$に対し,$\video \in \videoSet$は口唇動画, $\spWaveformGt \in \spWaveformSet$は原音声の音声波形,$\spkEmb \in \spkEmbSet$は話者ベクトル,$\melGt \in \melSet$はメルスペクトログラム,$\hubertIntGt \in \hubertIntSet$はHuBERT中間特徴量,$\hubertDiscGt \in \hubertDiscGtSet$はHuBERT離散特徴量とする.ここで,話者ベクトル$\spkEmb$は,話者識別モデル\cite{wan2018generalized}を利用して得られた音声の話者性を反映するベクトル(d-vector)であり,
\begin{equation}
    \spkEmb = \frac{1}{|\mathcal{\indexUpper}|} \sum_{\indexLower \in \mathcal{\indexUpper}} \spkEmbExtractor\lr{\bm{\outputLower}^{\text{sp-wf}}_{\indexLower}; \weightSpk}
\end{equation}
で与えられる.ここで,$\weightSpk$は事前学習済み重みを表す.$\mathcal{\indexUpper}$は学習データセット$\datasetTrain$において、話者$s_{n}$と話者が同一であるデータのインデックス集合から、ランダムに$\numUpper_{\text{spk-emb}}$個のインデックスを抽出した部分集合とする。すなわち、
\begin{equation}
    \mathcal{\indexUpper} \subset \{\indexLower \mid \indexLower \in \{1, \ldots, N\}, ~ s_{\indexLower} = \spkId\}, \quad |\mathcal{\indexUpper}| = \numUpper_{\text{spk-emb}}
\end{equation}
である。次に,HuBERT中間特徴量とHuBERT離散特徴量は,HuBERTにおける畳み込みエンコーダまで(Transformer層以前)を$\hubertConv$,HuBERTにおけるTransformer層($\hubertConv$以後)を$\hubertTransformer$,k-means法を$\kmeans$,インデックス系列をOne-hotベクトルに変換する関数を$\onehot$とすると,
\begin{gather}
    \hubertIntGt = \hubertConv\lr{\spWaveformGt; \weightHuBERTConv} \\
    \hubertDiscGt = \onehot\lr{\kmeans\lr{\hubertTransformer\lr{\hubertIntGt; \weightHuBERTTrans}}}
\end{gather}
で与えられる.ここで,$\weightHuBERTConv, \weightHuBERTTrans$はHuBERTの事前学習済み重みを表す.HuBERTを利用した特徴量計算の流れを図\ref{sec4:fig:hubert}に示す。

\begin{figure}[b]
    \centering
    \includegraphics[height=12mm]{./figure/sec4/model_2/hubert.drawio.png}
    \caption{HuBERTを利用した特徴量計算の流れ}
    \label{sec4:fig:hubert}
\end{figure}

提案手法を図\ref{sec4:fig:overview}に示す.提案手法は,ネットワークA,ネットワークB,ボコーダの三つからなる.まず,ネットワークAを$\myNetworkA$とすると,$\myNetworkA$の行う処理は,
\begin{equation}
    \label{sec4:eq:networkA_overview}
    \hubertIntPred, \melPredA, \hubertDiscPredA = \myNetworkA\lr{\video, \spkEmb; \weightA}
\end{equation}
と表される.ここで,$\hubertIntPred \in \hubertIntSet$は予測HuBERT中間特徴量,$\melPredA \in \melSet$はネットワークAの予測メルスペクトログラム,$\hubertDiscPredA \in \hubertDiscPredSet$はネットワークAのHuBERT離散特徴量に対するロジットを表す.次に,ネットワークBを$\myNetworkB$とすると,$\myNetworkB$の行う処理は,
\begin{equation}
    \melPredB, \hubertDiscPredB = \myNetworkB\lr{\hubertIntPred, \spkEmb; \weightB}
\end{equation}
と表される.ここで,$\melPredB \in \melSet$はネットワークBの予測メルスペクトログラム,$\hubertDiscPredB \in \hubertDiscPredSet$はネットワークBのHuBERT離散特徴量に対するロジットを表す.最後に,音声波形を生成するボコーダを$\vocoder$とすると,$\vocoder$の行う処理は,
\begin{equation}
    \spWaveformPred = \vocoder\lr{\melPredB, \hubertDiscPredB; \weightVoc}
\end{equation}
と表される.まとめると,提案手法は口唇動画と話者ベクトルを入力とし,ネットワークAとネットワークBによって中間表現を獲得後,中間表現をボコーダに入力することで音声波形を生成するものである.

\begin{figure}[bt]
    \centering
    \includegraphics[height=100mm]{./figure/sec4/model_2/overview.drawio.png}
    \caption{提案手法の全体像}
    \label{sec4:fig:overview}
\end{figure}

\subsubsection{ネットワークA}
ネットワークAを図\ref{sec4:fig:networkA}に示す。ネットワークAでは,まず,口唇動画$\video$をAVHuBERTに通すことで,特徴量$\featureA \in \featureASet$に変換する.これは,
\begin{equation}
    \featureA = \AVHuBERT\lr{\video; \weightAAVHuBERT}
\end{equation}
と表される.$\weightAAVHuBERT$は,AVHuBERTの事前学習済み重みで初期化した.次に,$\featureA$の各時刻$t$におけるベクトルに対し,話者ベクトル$\spkEmb$を次元方向に結合してから,全結合層によって次元を再度圧縮し,元の次元に戻す.これは,
\begin{equation}
    \featureA = \myNetworkSpkMerge\lr{\concat{\featureA, \lrRepeat{\spkEmb}{T}}; \weightAFcSpk}
\end{equation}
と表される.次に,話者ベクトルが統合された特徴量に対する後処理として,$\myNetworkPost$を適用する.これは,
\begin{equation}
    \featureA = \myNetworkPost\lr{\featureA; \weightAPost}
\end{equation}
と表される.ここで、後処理層$\myNetworkPost$の構造を図\ref{sec4:fig:post_three_step}に示す。$\myNetworkPost$は,一次元畳み込み層を主としたConvBlockおよび,これに残差結合を組み合わせたResBlockから構成される.$\myNetworkPost$は,話者ベクトル$\spkEmb$が次元方向に結合された特徴量$\featureA$に対する話者性を考慮した特徴量の変換を行うために導入した。$\AVHuBERT$はMasked Predictionによって事前学習されたモデルだから、動画の文脈的な構造を考慮するのに適していると考えられる一方で、音声合成に必要となる話者性は発話に依存しない、すなわち動画の文脈的な構造とは別の性質を持った情報だと考えたからである.最後に,$\featureA$を全結合層を通して変換することで,予測対象であるHuBERT中間特徴量,メルスペクトログラム,HuBERT離散特徴量に対するロジットを得る.これは,
\begin{gather}
    \hubertIntPred = \myNetworkFcHubInt\lr{\featureA; \weightAFcHubInt} \\
    \melPredA = \myNetworkFcMel\lr{\featureA; \weightAFcMel} \\
    \hubertDiscPredA = \myNetworkFcHubDisc\lr{\featureA; \weightAFcHuBDisc}
\end{gather}
と表される.$\weightAFcSpk, \weightAPost, \weightAFcHubInt, \weightAFcMel, \weightAFcHuBDisc$は,すべてランダムに初期化した.

ネットワークAの役割は,続くネットワークBの入力であるHuBERT中間特徴量を予測することである.これに対し,メルスペクトログラムとHuBERT離散特徴量の推定を同時に行った理由は,先行研究\cite{kim2023lip_multitask,choi2023intelligible}においてマルチタスク学習の有効性が確認されており、HuBERT中間特徴量の推定においても有効ではないかと考えたからである。

\begin{figure}[tb]
    \centering
    \begin{subfigure}[b]{0.48\textwidth}
        \centering
        \includegraphics[height=120mm]{./figure/sec4/model_2/networkA.drawio.png}
        \caption{ネットワークA}
        \label{sec4:fig:networkA}
    \end{subfigure}
    \hfill
    \begin{subfigure}[b]{0.48\textwidth}
        \centering
        \includegraphics[height=120mm]{./figure/sec4/model_2/networkB.drawio.png}
        \caption{ネットワークB}
        \label{sec4:fig:networkB}
    \end{subfigure}
    \hfill
    \caption{ネットワークAとネットワークBの構造}
    \label{sec4:fig:networkAB}
\end{figure}

\begin{figure}[tb]
    \centering
    \begin{subfigure}[b]{0.32\textwidth}
        \centering
        \includegraphics[height=70mm]{./figure/sec4/model_2/post.drawio.png}
        \caption{全体}
        \label{sec4:fig:post}
    \end{subfigure}
    \hfill
    \begin{subfigure}[b]{0.32\textwidth}
        \centering
        \includegraphics[height=70mm]{./figure/sec4/model_2/post_resblock.drawio.png}
        \caption{$\text{ResBlock}$}
        \label{sec4:fig:post_resblock}
    \end{subfigure}
    \hfill
    \begin{subfigure}[b]{0.32\textwidth}
        \centering
        \includegraphics[height=70mm]{./figure/sec4/model_2/post_convblock.drawio.png}
        \caption{$\text{ConvBlock}$}
        \label{sec4:fig:post_convblock}
    \end{subfigure}
    \caption{後処理層$\myNetworkPost$の構造}
    \label{sec4:fig:post_three_step}
\end{figure}

\subsubsection{ネットワークB}
ネットワークBを図\ref{sec4:fig:networkB}に示す。ネットワークBでは,まず,ネットワークAで得られた予測HuBERT中間特徴量$\hubertIntPred$を$\hubertTransformer$に通すことで,特徴量$\featureB \in \featureBSet$に変換する.これは,
\begin{equation}
    \featureB = \hubertTransformer\lr{\hubertIntPred; \weightBHuBERTTrans}
\end{equation}
と表される.$\weightBHuBERTTrans$は,HuBERTの事前学習済み重みで初期化する場合と,ランダム初期化する場合の二つを検討した.以下,ネットワークAと処理は同様であるため,数式のみ記載する.
\begin{gather}
    \featureB = \myNetworkSpkMerge\lr{\concat{\featureB, \lrRepeat{\spkEmb}{T}}; \weightBFcSpk} \\
    \featureB = \myNetworkPost\lr{\featureB; \weightBPost} \\
    \melPredB = \myNetworkFcMel\lr{\featureB; \weightBFcMel} \\
    \hubertDiscPredB = \myNetworkFcHubDisc\lr{\featureB; \weightBFcHuBDisc}
\end{gather}
$\weightBFcSpk, \weightBPost, \weightBFcMel, \weightBFcHuBDisc$は,すべてランダムに初期化した.

ネットワークBの役割は,メルスペクトログラムとHuBERT離散特徴量の予測である.HuBERT Transformer層の転移学習を検討した狙いは、HuBERTの自己教師あり学習の方法に基づく。HuBERTはMasked Predictionという自己教師あり学習を行っており、これは、畳み込みエンコーダ$\hubertConv$からの出力にマスクを適用し,Transformer層$\hubertTransformer$を通すことによって、マスクされたフレームにおける教師ラベルを推定する学習方法である.\ref{sec3:sec:ssl}節より、Masked Predictionでは欠損された入力から特徴抽出を行う必要があるから,最適化されたHuBERTは音声の文脈的な構造を学習していると考えられる.また、Masked Predictionは音声データのみによる学習が可能であり、近年では大規模なデータによって学習されたモデルも一般に公開されている。よって、本研究ではHuBERT Transformer層を動画音声合成にFine Tuningすることにより,動画を入力とするネットワークAの推定残差を,大量の音声データから学習された音声自体の文脈を考慮する力によって軽減することで、最終予測値であるメルスペクトログラムとHuBERT離散特徴量に対する推定精度改善を狙った.

\subsubsection{ボコーダ}
ボコーダを図\ref{sec4:fig:vocoder}に示す。本実験で用いるボコーダは、今回ベースラインとする先行研究\cite{choi2023intelligible}で提案されたMulti-input Vocoderを参考にしたものである。まず,ネットワークBで得られた予測メルスペクトログラム$\melPredB$とHuBERT離散特徴量に対するロジット$\hubertDiscPredB$を前処理層に通し,扱いやすい形状に変換する.これは,
\begin{gather}
    \featureVocMel = \vocoderPreMel\lr{\melPredB; \weightVocPreMel} \\
    \featureVocHuBERT = \vocoderPreHub\lr{\hubertDiscPredB; \weightVocPreHuB}
\end{gather}
と表される.$\featureVocMel \in \featureVocMelSet$は,$\melPredB$の時間方向に隣接したベクトルを次元方向に結合することで$\hubertDiscPredB$と系列長を揃え、その後全結合層を適用した特徴量である.一方,$\featureVocHuBERT \in \featureVocHuBERTSet$は,$\hubertDiscPredB$に対して$\argmax$関数を適用した後,各時刻$t$において選択されたインデックスをベクトルに変換した特徴量である.次に,$\featureVocMel, \featureVocHuBERT$を入力として,音声波形を生成する.これは,
\begin{equation}
    \spWaveformPred = \vocoderMain\lr{\concat{\featureVocMel, \featureVocHuBERT}; \weightVocMain}
\end{equation}
と表される.$\vocoderMain$は,図\ref{sec4:fig:vocoder_main}に示すように,一次元畳み込み層とUpsamplingBlockから構成され,これはHiFi-GANのGeneratorと同様である.各UpsamplingBlockでは,初めに一次元転置畳み込み層を通すことで時間方向のアップサンプリングを行い,その後複数の一次元畳み込み層から特徴抽出を行った結果を平均して出力する.

\begin{figure}[tb]
    \centering
    \begin{subfigure}[b]{0.32\textwidth}
        \centering
        \includegraphics[width=45mm]{./figure/sec4/model_2/vocoder.drawio.png}
        \caption{全体}
        \label{sec4:fig:vocoder_overview}
    \end{subfigure}
    \hfill
    \begin{subfigure}[b]{0.32\textwidth}
        \centering
        \includegraphics[width=45mm]{./figure/sec4/model_2/vocoder_main.drawio.png}
        \caption{$\vocoderMain$}
        \label{sec4:fig:vocoder_main}
    \end{subfigure}
    \hfill
    \begin{subfigure}[b]{0.32\textwidth}
        \centering
        \includegraphics[width=45mm]{./figure/sec4/model_2/vocoder_main_block.drawio.png}
        \caption{$\text{UpsamplingBlock}$}
        \label{sec4:fig:vocoder_main_block}
    \end{subfigure}
    \caption{ボコーダの構造}
    \label{sec4:fig:vocoder}
\end{figure}

\subsubsection{損失関数}
まず,ネットワークAの学習に用いる損失関数$\lossA$は,
\begin{equation}
    \begin{aligned}
        \lossA & \lr{\hubertIntGt, \hubertIntPred, \melGt, \melPredA, \hubertDiscGt, \hubertDiscPredA}                 \\
        =      & \lossWeightHubInt \lossMAE{\hubertIntGt}{\hubertIntPred} + \lossWeightMel \lossMAE{\melGt}{\melPredA} \\
               & + \lossWeightHubDisc \lossCE{\hubertDiscGt}{\hubertDiscPredA}
    \end{aligned}
\end{equation}
で与えられる.すなわち,HuBERT中間特徴量についてのMAE Loss,メルスペクトログラムについてのMAE Loss,HuBERT離散特徴量についてのCross Entropy Lossの重み付け和である.次に,ネットワークBの学習に用いる損失関数$\lossB$は,
\begin{equation}
    \begin{aligned}
        \lossB & \lr{\melGt, \melPredB, \hubertDiscGt, \hubertDiscPredB}                                                  \\
        =      & \lossWeightMel \lossMAE{\melGt}{\melPredB} + \lossWeightHubDisc \lossCE{\hubertDiscGt}{\hubertDiscPredB}
    \end{aligned}
\end{equation}
で与えられる.すなわち,メルスペクトログラムについてのMAE Loss,HuBERT離散特徴量についてのCross Entropy Lossの重み付け和である.最後に,ボコーダの学習に用いる損失関数について,これはHiFi-GANと同様であるから,ここでは割愛する.
