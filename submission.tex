\documentclass[12pt]{jarticle}
\usepackage[utf8]{inputenc}
\usepackage[top=30truemm, bottom=30truemm, left=20truemm, right=20truemm]{geometry}
\usepackage{amsmath}
\usepackage{amssymb}
\usepackage{mathtools}
\usepackage{siunitx}
\usepackage{bm}
\usepackage[dvipdfmx]{graphicx}
\usepackage[dvipdfmx]{color}
\usepackage[subrefformat=parens]{subcaption}
\usepackage{booktabs}
\usepackage{tabularx}
\usepackage{tocloft}
\usepackage{enumerate}
\usepackage{url}
\usepackage{multirow}
\usepackage{caption}
\usepackage{enumitem}
\usepackage{makecell}
\usepackage{array}
\usepackage{algorithm}
\usepackage{algpseudocode}
\usepackage{physics}
\usepackage{hyperref}

\numberwithin{equation}{section}    % 数式番号にセクション番号をつける
\numberwithin{figure}{section}      % 図番号にセクション
\numberwithin{table}{section}      % 図番号にセクション

% \renewcommand{\figurename}{Fig.}    % 図 -> Fig.
% \renewcommand{\tablename}{Table }   % 表 -> Table

\renewcommand{\baselinestretch}{1.1}

\newlength{\figcaptionskip}
\setlength{\figcaptionskip}{5pt} % 図のキャプション間隔
\newlength{\tabcaptionskip}
\setlength{\tabcaptionskip}{-5pt} % 表のキャプション間隔
\captionsetup[figure]{skip=\figcaptionskip}
\captionsetup[table]{skip=\tabcaptionskip}

\setlist[enumerate]{topsep=5pt, partopsep=5pt, itemsep=0pt, parsep=0pt}
\setlength{\topsep}{5pt}
\setlength{\partopsep}{5pt}

\DeclareSIUnit{\fps}{fps}
\DeclareSIUnit{\dBFS}{dBFS}

\captionsetup[subfigure]{labelformat=simple}
\renewcommand{\thesubfigure}{(\alph{subfigure})}

% section3
% data
\newcommand{\dataset}{\mathcal{D}}
\newcommand{\datasetTrain}{\dataset_{\text{train}}}
\newcommand{\dimLower}{d}
\newcommand{\dimUpper}{D}
\newcommand{\freqLower}{f}
\newcommand{\freqUpper}{F}
\newcommand{\heightLower}{h}
\newcommand{\heightUpper}{H}
\newcommand{\indexLower}{i}
\newcommand{\indexUpper}{I}
\newcommand{\inputLower}{x}
\newcommand{\inputUpper}{X}
\newcommand{\imUnit}{j}
\newcommand{\melLower}{m}
\newcommand{\melUpper}{M}
\newcommand{\numLower}{n}
\newcommand{\numUpper}{N}
\newcommand{\outputLower}{y}
\newcommand{\outputUpper}{Y}
\newcommand{\timeLower}{t}
\newcommand{\timeUpper}{T}
\newcommand{\widthLower}{w}
\newcommand{\widthUpper}{W}
\newcommand{\std}{\sigma}
\newcommand{\mean}{\mu}
% model
\newcommand{\biasLower}{b}
\newcommand{\classLower}{c}
\newcommand{\classUpper}{C}
\newcommand{\dilationUpper}{R}
\newcommand{\fcLayer}{\text{FC}}
\newcommand{\fcLayerValue}[2]{\text{FC}\lr{#1; #2}}
\newcommand{\numHeadLower}{h}
\newcommand{\numHeadUpper}{H}
\newcommand{\kernelSizeLower}{k}
\newcommand{\kernelSizeUpper}{K}
\newcommand{\modelLower}{f}
\newcommand{\modelUpper}{F}
\newcommand{\strideUpper}{S}
\newcommand{\weightLower}{w}
\newcommand{\weightUpper}{W}
\newcommand{\weightAndBias}{\theta}
% other
\newcommand{\binaryMaskLower}{m}
\newcommand{\binaryMaskUpper}{M}
\newcommand{\optimEmaConst}{\beta}
\newcommand{\iter}{\tau}
\newcommand{\epoch}{\nu}
\newcommand{\learningRate}{\eta}
\newcommand{\lossFuncUpper}{L}
\newcommand{\normScale}{\gamma}
\newcommand{\normShift}{\beta}
\newcommand{\probaLower}{p}
\newcommand{\regConst}{\lambda}
% math
\newcommand{\realSet}{\mathbb{R}}
\newcommand{\naturalSet}{\mathbb{N}}
\newcommand{\complexSet}{\mathbb{C}}
% 囲み
\newcommand{\lrClosedInterval}[2]{\left[#1, #2\right]}
\newcommand{\lrOpenInterval}[2]{\left(#1, #2\right)}
\newcommand{\lrLeftClosedInterval}[2]{\left[#1, #2\right)}
\newcommand{\lrRightClosedInterval}[2]{\left(#1, #2\right]}
\newcommand{\lr}[1]{\left(#1\right)}
\newcommand{\lrsq}[1]{\left[#1\right]}
\newcommand{\lrc}[1]{\left\{#1\right\}}
\newcommand{\lrFloor}[1]{\left\lfloor#1\right\rfloor}
\newcommand{\lrAbs}[1]{\left|#1\right|}
\newcommand{\lrOneNorm}[1]{\left\|#1\right\|_{1}}
\newcommand{\lrTwoNorm}[1]{\left\|#1\right\|_{2}}
\newcommand{\lrRepeat}[2]{\left(#1\right)_{#2}}
\newcommand{\lrRepeatTr}[2]{\lr{\lr{#1}_{#2}}^\top}
\newcommand{\card}[1]{\left|#1\right|}

% operator
\newcommand{\expectation}[1]{\mathbb{E}\lrsq{#1}}
\newcommand{\elemMul}{\odot}
\newcommand{\concat}[1]{\lrsq{#1}}
\newcommand{\onehot}{\text{one-hot}}
\newcommand{\argmax}{\text{argmax}}
\newcommand{\softmax}{\text{softmax}}
\newcommand{\sigmoid}{\sigma}
\newcommand{\relu}{\text{ReLU}}
\newcommand{\leakyRelu}{\text{LeakyReLU}}
\newcommand{\prelu}{\text{PReLU}}
\newcommand{\gelu}{\text{GELU}}

% section4
% model
\newcommand{\spkEmbExtractor}{f_{\text{spk}}}
\newcommand{\hubert}{f_{\text{HuB}}}
\newcommand{\hubertConv}{f_{\text{HuB-conv}}}
\newcommand{\hubertTransformer}{f_{\text{HuB-trans}}}
\newcommand{\kmeans}{f_{\text{k-means}}}
\newcommand{\AVHuBERT}{f_{\text{AVHuB}}}
\newcommand{\myNetworkA}{f_{\text{A}}}
\newcommand{\myNetworkB}{f_{\text{B}}}
\newcommand{\myNetworkSpkMerge}{\text{FC}}
\newcommand{\myNetworkPost}{f_{\text{post}}}
\newcommand{\myNetworkFcMel}{\text{FC}}
\newcommand{\myNetworkFcHubDisc}{\text{FC}}
\newcommand{\myNetworkFcHubInt}{\text{FC}}
\newcommand{\vocoder}{f_{\text{voc}}}
\newcommand{\vocoderPreMel}{f_{\text{voc-pre-mel}}}
\newcommand{\vocoderPreHub}{f_{\text{voc-pre-HuB}}}
\newcommand{\vocoderMain}{f_{\text{voc-main}}}
\newcommand{\mpd}{f_{\text{mpd}}}
\newcommand{\msd}{f_{\text{msd}}}
\newcommand{\weightSpk}{\bm{\weightAndBias}_{\text{spk}}}
\newcommand{\weightHuBERT}{\bm{\weightAndBias}_{\text{HuB}}}
\newcommand{\weightHuBERTConv}{\bm{\weightAndBias}_{\text{HuB-conv}}}
\newcommand{\weightHuBERTTrans}{\bm{\weightAndBias}_{\text{HuB-trans}}}
\newcommand{\weightA}{\bm{\weightAndBias}_{\text{A}}}
\newcommand{\weightAAVHuBERT}{\bm{\weightAndBias}_{\text{A-AVHuB}}}
\newcommand{\weightAFcSpk}{\bm{\weightAndBias}_{\text{A-fc-spk}}}
\newcommand{\weightAPost}{\bm{\weightAndBias}_{\text{A-post}}}
\newcommand{\weightAFcHubInt}{\bm{\weightAndBias}_{\text{A-fc-HuB-int}}}
\newcommand{\weightAFcMel}{\bm{\weightAndBias}_{\text{A-fc-mel}}}
\newcommand{\weightAFcHuBDisc}{\bm{\weightAndBias}_{\text{A-fc-HuB-Disc}}}
\newcommand{\weightB}{\bm{\weightAndBias}_{\text{B}}}
\newcommand{\weightBHuBERTTrans}{\bm{\weightAndBias}_{\text{B-HuB-trans}}}
\newcommand{\weightBFcSpk}{\bm{\weightAndBias}_{\text{B-fc-spk}}}
\newcommand{\weightBPost}{\bm{\weightAndBias}_{\text{B-post}}}
\newcommand{\weightBFcMel}{\bm{\weightAndBias}_{\text{B-fc-mel}}}
\newcommand{\weightBFcHuBDisc}{\bm{\weightAndBias}_{\text{B-fc-HuB-disc}}}
\newcommand{\weightVoc}{\bm{\weightAndBias}_{\text{voc}}}
\newcommand{\weightVocPreMel}{\bm{\weightAndBias}_{\text{voc-pre-mel}}}
\newcommand{\weightVocPreHuB}{\bm{\weightAndBias}_{\text{voc-pre-HuB}}}
\newcommand{\weightVocMain}{\bm{\weightAndBias}_{\text{voc-main}}}
% data
\newcommand{\featureUpper}{H}
\newcommand{\spkId}{s_{\numLower}}
\newcommand{\spkIdSet}{\{1, \ldots, \numUpper_{\text{spk}}\}}
\newcommand{\spWaveformGt}{\bm{\outputLower}^{\text{sp-wf}}_{\numLower}}
\newcommand{\spWaveformPred}{\hat{\bm{\outputLower}}^{\text{sp-wf}}_{\numLower}}
\newcommand{\spWaveformSet}{\realSet^{\timeUpper^{\text{sp-wf}}_{\numLower}}}
\newcommand{\spkEmb}{\bm{\inputLower}^{\text{spk-emb}}_{\spkId}}
\newcommand{\spkEmbSet}{\realSet^{\dimUpper^{\text{spk-emb}}}}
\newcommand{\hubertIntGt}{\bm{\outputUpper}^{\text{HuB-int}}_{\numLower}}
\newcommand{\hubertIntPred}{\hat{\bm{\outputUpper}}^{\text{HuB-int}}_{\numLower}}
\newcommand{\hubertIntSet}{\realSet^{\timeUpper^{\text{HuB}}_{\numLower} \times \dimUpper^{\text{HuB-int}}}}
\newcommand{\hubertDiscGt}{\bm{\outputUpper}^{\text{HuB-disc}}_{\numLower}}
\newcommand{\hubertDiscGtVec}{\bm{\outputLower}^{\text{HuB-disc}}_{\numLower, \timeLower}}
\newcommand{\hubertDiscGtSet}{\{0, 1\}^{\timeUpper^{\text{HuB}}_{\numLower} \times \classUpper}}
\newcommand{\hubertDiscPredA}{\hat{\bm{\outputUpper}}^{\text{HuB-disc-A}}_{\numLower}}
\newcommand{\hubertDiscPredB}{\hat{\bm{\outputUpper}}^{\text{HuB-disc-B}}_{\numLower}}
\newcommand{\hubertDiscPredSet}{\realSet^{\timeUpper^{\text{HuB}}_{\numLower} \times \classUpper}}
\newcommand{\melGt}{\bm{\outputUpper}^{\text{mel}}_{\numLower}}
\newcommand{\melPredA}{\hat{\bm{\outputUpper}}^{\text{mel-A}}_{\numLower}}
\newcommand{\melPredB}{\hat{\bm{\outputUpper}}^{\text{mel-B}}_{\numLower}}
\newcommand{\melSet}{\realSet^{\timeUpper^{\text{mel}}_{\numLower} \times \dimUpper^{\text{mel}}}}
\newcommand{\video}{\bm{\inputUpper}^{\text{video}}_{\numLower}}
\newcommand{\videoSet}{\realSet^{\timeUpper^{\text{video}}_{\numLower} \times \widthUpper \times \heightUpper}}
\newcommand{\featureA}{\bm{\featureUpper}^{\text{A}}_{\numLower}}
\newcommand{\featureASet}{\realSet^{\timeUpper^{\text{video}}_{\numLower} \times \dimUpper^{\text{A}}}}
\newcommand{\featureB}{\bm{\featureUpper}^{\text{B}}_{\numLower}}
\newcommand{\featureBSet}{\realSet^{\timeUpper^{\text{HuB}}_{\numLower} \times \dimUpper^{\text{B}}}}
\newcommand{\featureVocMel}{\bm{\featureUpper}^{\text{mel-voc}}_{\numLower}}
\newcommand{\featureVocMelSet}{\realSet^{\timeUpper^{\text{HuB}}_{\numLower} \times \dimUpper^{\text{mel-voc}}}}
\newcommand{\featureVocHuBERT}{\bm{\featureUpper}^{\text{HuB-voc}}_{\numLower}}
\newcommand{\featureVocHuBERTSet}{\realSet^{\timeUpper^{\text{HuB}}_{\numLower} \times \dimUpper^{\text{HuB-voc}}}}
% loss
\newcommand{\lossWeightMel}{\lambda_{\text{mel}}}
\newcommand{\lossWeightHubInt}{\lambda_{\text{HuB-int}}}
\newcommand{\lossWeightHubDisc}{\lambda_{\text{HuB-disc}}}
\newcommand{\lossA}{\lossFuncUpper_{A}}
\newcommand{\lossB}{\lossFuncUpper_{B}}
\newcommand{\lossMAE}[2]{\lossFuncUpper_{\text{MAE}}\lr{#1, #2}}
\newcommand{\lossCE}[2]{\lossFuncUpper_{\text{CE}}\lr{#1, #2}}

\allowdisplaybreaks[4]

\begin{document}

\begin{titlepage}
    \begin{center}
        {\Large 2024年度後期 修士論文}
        \vspace{120truept}

        {\huge 深層学習による口唇動画を用いた音声合成に関する研究}
        \vspace{30truept}

        {\huge A Study on Speech Synthesis from Lip Video Using Deep Learning}
        \vspace{120truept}

        {\Large 2025年1月15日}
        \vspace{10truept}

        {\Large 九州大学芸術工学府音響設計コース}
        \vspace{70truept}

        {\Large 2DS23095M}
        \vspace{10truept}

        {\Large 南 汰翼}
        \vspace{10truept}

        {\Large MINAMI Taisuke}
        \vspace{30truept}

        {\Large 研究指導教員 鏑木 時彦 教授}
    \end{center}
\end{titlepage}

\section*{概要}
\thispagestyle{empty}
癌などの病気で喉頭を摘出した場合,元通りに発声することが不可能になる.これに対し,現状でもいくつかの代用音声が存在するが,これらにもそれぞれデメリットがある.そこで本研究では,新たな代用音声として,ビデオカメラで撮影された口唇動画を入力として音声を合成する,深層学習モデルを利用したアプローチを検討する.動画音声合成の分野では,これまでにも多くの研究によって改善が重ねられてきた.近年では,複数の予測対象を用いたマルチタスク学習や,自己教師あり学習モデルの転移学習の有効性が確認されている.しかし,動画音声合成モデルによって得られる合成音声の品質は依然として低く,自然音声に迫る合成音の実現に至っていない点が課題である.これに対し,本研究では,近年高い精度を達成した「AVHuBERTを利用したメルスペクトログラムとHuBERT離散特徴量を予測対象とするマルチタスク学習手法」をベースラインとして採用し,この手法を上回る新たなモデルを提案することで,自然音声に迫る合成音声の実現に近づくことを目的とした.ここで,動画音声合成を困難にしている一因として,同じような口唇の動きでも,場合によって異なる音素となる場合があることが挙げられる.これに対し,提案手法は自己教師あり学習モデルであるHuBERTを導入し,動画から予測された不完全な予測結果を,大量の音声データを元に学習された文脈を考慮する能力によって補完することで,最終予測値に対する予測精度改善を狙った.実験では,合成音声の明瞭性および,合成音声と原音声の類似性を,客観評価と主観評価の両面から評価した.その結果,提案手法におけるHuBERT Transformerの転移学習は仮説に反して有効ではなかったものの,HuBERT Transformerをランダム初期化した場合が有効性を示し,客観評価と主観評価の両面においてベースラインを上回った.また,本実験を通して明らかとなった課題から,さらなる改善の可能性があることも示唆された.
\clearpage

\setcounter{tocdepth}{2}
\tableofcontents
\thispagestyle{empty}
\clearpage

\pagestyle{plain}
\setcounter{page}{1}

\section{序論}
\subsection{背景}
音声は基本的なコミュニケーション手段として,人々の日常生活において重要な役割を果たしている.しかし、癌などの病気で喉頭を摘出すると、声帯振動による音声生成が不可能になり、従来の発声手段を失ってしまう。このような場合の代用音声手法として、電気式人工喉頭や食道発声、シャント発声がある。しかし、これらには人工的な音声になる、習得に訓練が必要である、器具の交換のため定期的な手術を要するなどの課題がある。そこで本研究では,新たな代用音声手法として,ビデオカメラで撮影された口唇動画を入力として音声を合成する、深層学習モデルを利用したアプローチを提案する。この手法により,訓練や手術を必要とせずとも,自然な声でのコミュニケーションを可能にすることを目指す.

従来の動画音声合成では,動画からの予測対象にメルスペクトログラムが選択されることが多かった.例えば,テキスト音声合成で高い品質を達成した系列変換モデル\cite{shen2018natural}を動画音声合成に応用したLip2Wav\cite{prajwal2020learning}や,GAN(Generative Adversarial Network)を活用したVCA-GAN\cite{kim2021lip},畳み込み層とTransformer\cite{vaswani2017attention}から構成されるConformer\cite{gulati2020conformer}を利用したSVTS\cite{mira2022svts}などがある.これらでは,損失関数としてMAE LossやMSE Lossが用いられ,原音声から計算されるメルスペクトログラムと,予測されたメルスペクトログラムの間の距離を最小化するよう学習が行われた.これに対し,\cite{kim2023lip_multitask}では,メルスペクトログラムは発話内容だけでなく、話者性やイントネーションといった音響的特徴を含むため、発話内容に対する正確な制約を十分に与えられないことを課題として指摘した.この課題に対し,モデルの中間特徴量を用いたテキストの予測によって計算されるCTC(Connectionist Temporal Classification) Lossと,メルスペクトログラムを事前学習済み音声認識モデルによって変換した特徴量におけるMSE Lossを損失に加え,発話内容に関する制約を強化した.これにより,客観評価において従来手法を上回る結果が得られた.これに続き,\cite{choi2023intelligible}では手法\cite{kim2023lip_multitask}が教師ラベルとしてテキストを必要とするため,テキストアノテーションされていないデータで用いることができない点を課題として指摘した.この課題に対し,言語情報を中心とした特徴量として扱うことが可能な,HuBERT\cite{hsu2021hubert}を利用して得られた離散特徴量をテキストに代わる予測対象として導入した.また,モデルから予測されるメルスペクトログラムがノイジーになってしまう課題に対し,言語情報であるHuBERT離散特徴量も入力として情報を補完する,Multi-input Vocoderが合わせて提案された.これにより,客観評価と主観評価の両面において従来手法を上回る結果が得られた.加えて,\cite{choi2023intelligible}ではAVHuBERT\cite{shi2022learning}のFineTuningも検討された.AVHuBERTは,英語動画音声データセットであるLRS3\cite{afouras2018lrs3}と,多言語動画音声データセットであるVoxCeleb2\cite{chung2018voxceleb2}の英語データを用いて,動画と音声の間の複雑な関係をMasked Predictionという自己教師あり学習方法によって学習したモデルである.動画音声合成に対するFineTuningの結果,AVHuBERTを用いない場合と比較して,客観評価指標における改善が確認された.

\subsection{目的}
本研究では,動画音声合成モデルによって得られる合成音声の品質が依然として低く,自然音声に迫る合成音の実現に至っていない点を課題とする.この課題に対し,近年高い精度を達成した,「AVHuBERTを利用したメルスペクトログラムとHuBERT離散特徴量を予測対象とするマルチタスク学習手法」をベースラインとして採用し,この手法を上回る新たなモデルを提案することで、自然音声に迫る合成音声の実現に近づくことを目的とする。

ここで,近年有効性の示された,テキストやHuBERT離散特徴量を利用するマルチタスク学習手法は,動画を入力とするモデルの学習方法に対する工夫だったと考えられる.しかし,口唇動画と発話内容の間に一意な対応がないこと,すなわち,同じような口唇の動きでも場合によって異なる音素となる場合があることが,特に動画音声合成を困難にしていると考える.これに対して本研究では,動画を入力として最終予測値を出力する従来のネットワークとは異なる新たなネットワークを導入し,これらを組み合わせることで精度改善を狙った.

\subsection{本論文の構成}
本論文は本章を含め,全5章から構成される.2章では,音声データを取り扱う上で必要な信号処理について述べる.3章では,動画から音声を予測するために用いた深層学習について述べる.4章では,本研究の提案手法とベースラインとの比較について述べる.これを踏まえ,5章では本研究を通した結論を述べる.

\clearpage

\section{音声信号処理}
音声にはフォルマントや基本周波数(ピッチ)など,様々な周波数的な特徴が存在している.フォルマントは母音や子音を知覚するため,ピッチはアクセントやイントネーションを表現するために重要なものである.このような音声信号の持つ複雑さから,時間波形のままその特徴を分析することは困難である.これに対し,本節では音声の特徴を捉えやすくするための信号処理について説明する.

\subsection{音声のフーリエ変換}
音声の時間波形から周波数領域の情報を得るためには,フーリエ変換(Fourier transform)が用いられる.音声は通常マイクロフォンで収録され,コンピュータ内で処理される際には,アナログ信号ではなくデジタル信号として扱われる.このデジタル信号は,サンプリング周波数と量子化ビット数に従って離散化されている.離散化された信号に対しては,離散フーリエ変換(discrete Fourier transform; DFT)が適用される.また,信号の系列長をゼロパディングして2のべき乗の長さに調整すれば,計算量を抑えた高速フーリエ変換(fast Fourier transform; FFT)を用いることができる.

離散信号を$x\lrsq{\numLower}$,それに対するフーリエ変換を$X\lrsq{\freqLower}$とする.ここで,$\numLower$はサンプルのインデックス,$\freqLower$は周波数インデックスである.$X\lrsq{\freqLower}$は複素数であるから,以下のように極座標表示することができる.
\begin{align}
    X\lrsq{\freqLower} & = \Re\lr{X\lrsq{\freqLower}} + \imUnit\Im\lr{X\lrsq{\freqLower}}    \\
                       & = \lrAbs{X\lrsq{\freqLower}}\exp^{\imUnit\angle X\lrsq{\freqLower}}
\end{align}
ここで,$\lrAbs{X\lrsq{\freqLower}}$は振幅特性(振幅スペクトル),$\angle X\lrsq{\freqLower}$は位相特性(位相スペクトル)であり,
\begin{gather}
    \lrAbs{X\lrsq{\freqLower}} = \sqrt{\Re\lr{X\lrsq{\freqLower}}^{2} + \Im\lr{X\lrsq{\freqLower}}^{2}} \\
    \angle X\lrsq{\freqLower} = \tan^{-1} \frac{\Im\lr{X\lrsq{\freqLower}}}{\Re\lr{X\lrsq{\freqLower}}}
\end{gather}
と表される.また,$\lrAbs{X\lrsq{\freqLower}}^2$はパワースペクトルと呼ばれる.これにより,信号中にどのような周波数成分がどれくらい含まれているかを調べることができる.しかし,音声はフォルマントやピッチが時々刻々と変化するため,信号全体に対して直接フーリエ変換を適用したとしても有用な結果が得られない.このような音声の持つ非定常性の問題に対して,十分短い時間幅においては信号の定常性が成り立つという仮定のもと,短時間フーリエ変換(short-time Fourier transform; STFT)が用いられる.STFTでは,音声信号に対して窓関数による窓処理を適用し,短時間に区切られた信号それぞれに対してDFTを適用する.ここで,窓処理とはある特定の窓関数と音声信号を時間領域でかけ合わせることであり,窓関数の時間幅を窓長という.また,窓関数を時間方向にシフトするときの時間幅をシフト幅という.STFTには,時間分解能と周波数分解能の間にトレード・オフの関係がある.窓長が長い場合には周波数分解能が向上する一方,時間分解能が低下する.窓長が短い場合にはその逆となる.音声信号$x\lrsq{\numLower}$のSTFTを時刻$\timeLower$,周波数インデックスを$\freqLower$として$X\lrsq{\timeLower, \freqLower}$と表すと,$X\lrsq{\timeLower, \freqLower}$は時間周波数領域における複素数となる.これを複素スペクトログラムと呼ぶ.また,$|X\lrsq{\timeLower, \freqLower}|$を振幅スペクトログラム,$\angle X\lrsq{\timeLower, \freqLower}$を位相スペクトログラム,
$\lrAbs{X\lrsq{\timeLower, \freqLower}}^{2}$をパワースペクトログラムと呼ぶ.ここで,「小さな鰻屋に,熱気のようなものがみなぎる」と発話した,サンプリング周波数\SI{16}{\kHz}の音声波形に対し,窓関数としてハニング窓を用いた上で,複数の窓長・シフト幅によって計算した対数パワースペクトログラムを,図~\ref{sec2:fig:log_power_spectrograms}に示す.窓長が\SI{100}{\ms}と長い場合には周波数分解能が高いが,時間分解能が低下することでスペクトルの時間変化が滑らかでないことがわかる.一方,窓長が\SI{12.5}{\ms}と短い場合には時間分解能が高いが,周波数分解能が低下することでスペクトルがぼやけていることがわかる.これが窓長に対する時間分解能と周波数分解能のトレード・オフであり,窓長としては\SI{25}{\ms}や\SI{50}{ms}が適当なパラメータであることがわかる.
\begin{figure}[tb]
    \centering
    \begin{subfigure}[b]{0.48\textwidth}
        \centering
        \includegraphics[width=\textwidth]{./figure/sec2/spectrogram_1.png}
        \caption{窓長\SI{12.5}{\ms},シフト幅\SI{5}{\ms}}
        \label{sec2:fig:spectrogram1}
    \end{subfigure}
    \begin{subfigure}[b]{0.48\textwidth}
        \centering
        \includegraphics[width=\textwidth]{./figure/sec2/spectrogram_2.png}
        \caption{窓長\SI{25}{\ms},シフト幅\SI{10}{\ms}}
        \label{sec2:fig:spectrogram2}
    \end{subfigure}

    \vspace{0.5cm}

    \begin{subfigure}[b]{0.48\textwidth}
        \centering
        \includegraphics[width=\textwidth]{./figure/sec2/spectrogram_4.png}
        \caption{窓長\SI{50}{\ms},シフト幅\SI{20}{\ms}}
        \label{sec2:fig:spectrogram3}
    \end{subfigure}
    \begin{subfigure}[b]{0.48\textwidth}
        \centering
        \includegraphics[width=\textwidth]{./figure/sec2/spectrogram_8.png}
        \caption{窓長\SI{100}{\ms},シフト幅\SI{40}{\ms}}
        \label{sec2:fig:spectrogram4}
    \end{subfigure}
    \caption{「小さな鰻屋に,熱気のようなものがみなぎる」と発話した音声から計算された対数パワースペクトログラム}
    \label{sec2:fig:log_power_spectrograms}
\end{figure}

\subsection{メルスペクトログラム}
メルスペクトログラムは,パワースペクトログラムにメルフィルタバンクをかけることによって得られる音響特徴量であり,音声認識や音声合成といったタスクにおいて広く用いられている.メルフィルタバンクは,周波数軸を人間の聴感特性を考慮して変換したメル尺度上で,指定した数のフィルタを等間隔に配置して得られる.図\ref{sec2:fig:melfb}に,フィルタを配置する帯域を\SI{0}{\kHz}から\SI{8}{\kHz}とした場合におけるメルフィルタバンクを示す.中心周波数の低いフィルタほど帯域幅が狭くなっており,低域ほど細かく,高域ほど荒く周波数成分を扱っていることがわかる.また,フィルタ数を20から80に増やすことによって各フィルタの帯域幅が狭くなり,より細かく周波数成分を分析できることがわかる.例として,「小さな鰻屋に,熱気のようなものがみなぎる」と発話した,サンプリング周波数\SI{16}{\kHz}の音声波形に対し,窓長\SI{25}{\ms}のハニング窓を用いてシフト幅\SI{10}{\ms}でパワースペクトログラムを計算し,フィルタ数80のメルフィルタバンクを適用して得られた対数メルスペクトログラムを図~\ref{sec2:fig:melspectrogram}に示す.

\begin{figure}[tb]
    \centering
    \begin{subfigure}[b]{1.0\textwidth}
        \centering
        \includegraphics[width=150mm]{./figure/sec2/melfb/onedim_20.png}
        \caption{フィルタ数を20とした場合}
        \label{sec2:fig:melfb_20}
    \end{subfigure}

    \vspace{0.5cm}

    \begin{subfigure}[b]{1.0\textwidth}
        \centering
        \includegraphics[width=150mm]{./figure/sec2/melfb/onedim_80.png}
        \caption{フィルタ数を80とした場合}
        \label{sec2:fig:melfb_80}
    \end{subfigure}
    \caption{フィルタを配置する帯域を\SI{0}{\kHz}から\SI{8}{\kHz}とした場合におけるメルフィルタバンク}
    \label{sec2:fig:melfb}
\end{figure}

\begin{figure}[bt]
    \centering
    \includegraphics[height=60mm]{./figure/sec2/melspectrogram.png}
    \caption{「小さな鰻屋に,熱気のようなものがみなぎる」と発話した音声に対する対数メルスペクトログラム}
    \label{sec2:fig:melspectrogram}
\end{figure}
\clearpage

\section{深層学習}
深層学習とは,人間の神経細胞の仕組みを模擬したニューラルネットワークを用いる機械学習手法のことである.特に近年ではその層を深くしたディープニューラルネットワーク(Deep Neural Network; DNN)が用いられ,大量のパラメータによる表現力により,自然言語処理や画像処理,音声処理など様々な分野で成果をあげている.本章では,DNNの構成要素及び,構築したDNNの学習方法について説明する.
\subsection{DNNの構成要素}
\subsubsection{全結合層}
全結合層は,入力に対して線型変換を施す層である.全結合層への入力を$\bm{\inputLower} \in \realSet^{\dimUpper_{\text{in}}}$とすると,出力$\bm{\outputLower} \in \realSet^{\dimUpper_{\text{out}}}$は,
\begin{equation}
    \bm{\outputLower} = \fcLayerValue{\bm{\inputLower}}{\bm{\weightAndBias}} = \lr{\bm{\inputLower}^{\top} \bm{\weightUpper}}^{\top} + \bm{\biasLower}
\end{equation}
で与えられる.ここで,$\dimUpper_{\text{in}}, \dimUpper_{\text{out}}$は入出力の次元,$\bm{\weightUpper} \in \realSet^{\dimUpper_{\text{in}} \times \dimUpper_{\text{out}}}$は重み,$\bm{\biasLower} \in \realSet^{\dimUpper_{\text{out}}}$はバイアスであり,$\bm{\weightAndBias}$は重みとバイアスをまとめて表す変数とする.また,入力が行列$\bm{\inputUpper} \in \realSet^{\timeUpper \times \dimUpper_{\text{in}}}$である時,出力$\bm{Y} \in \realSet^{\timeUpper \times \dimUpper_{\text{out}}}$は,
\begin{equation}
    \bm{\outputUpper} = \fcLayerValue{\bm{\inputUpper}}{\bm{\weightAndBias}} = \bm{\inputUpper} \bm{\weightUpper} + \bm{\biasLower} \bm{1}^{\top}
\end{equation}
で与えられる.ここで,$\timeUpper$は系列長,$\bm{1} \in \{1\}^{\timeUpper}$は全成分が1のベクトルである.全結合層はDNN内部での特徴量の次元の変換や,最終層において所望の出力に次元を合わせるのに用いられる.

\subsubsection{畳み込み層}
畳み込み層は,入力に対して畳み込み演算を行う層である.1次元畳み込み層について,入力を$\bm{\inputUpper} \in \realSet^{\dimUpper_{\text{in}} \times \timeUpper_{\text{in}}}$,出力を$\bm{\outputUpper} \in \realSet^{\dimUpper_{\text{out}} \times \timeUpper_{\text{out}}}$とし,それぞれ$\dimLower$次元目,$\timeLower$番目の成分を$\inputLower_{\dimLower_{\text{in}}, \timeLower}, \outputLower_{\dimLower_{\text{out}}, \timeLower}$で表す.$\dimUpper_{\text{in}}, \dimUpper_{\text{out}}$は入出力の次元,$\timeUpper_{\text{in}}, \timeUpper_{\text{out}}$は入出力の系列長である.このとき,$\outputLower_{\dimLower_{\text{out}}, \timeLower}$は,
\begin{align}
    \outputLower_{\dimLower_{\text{out}}, \timeLower} = b_{\dimLower_{\text{out}}} + \sum_{\dimLower_{\text{in}} = 1}^{\dimUpper_{\text{in}}} \sum_{\kernelSizeLower = 1}^{\kernelSizeUpper} \inputLower_{\dimLower_{\text{in}}, \timeLower - \lrFloor{\frac{\kernelSizeUpper}{2}} + \kernelSizeLower - 1} \weightLower_{\dimLower_{\text{in}}, \dimLower_{\text{out}}, \kernelSizeLower}
\end{align}
で与えられる.ここで,$\kernelSizeUpper$はカーネルサイズ,$\weightLower_{\dimLower_{\text{in}}, \dimLower_{\text{out}}, \kernelSizeLower}$は入力の$\dimLower_{\text{in}}$次元目から出力の$\dimLower_{\text{out}}$次元目に割り当てられたカーネルの$\kernelSizeLower$番目の成分,$b_{\dimLower_{\text{out}}}$は出力の$\dimLower_{\text{out}}$次元目に割り当てられたバイアスである.上式より,1次元畳み込み層の$\timeLower$番目の出力は,$\timeLower$番目を中心としたカーネルサイズの範囲分の入力から計算されることがわかる($\kernelSizeUpper$は奇数を想定した).これより,畳み込み層は入力の局所的な特徴を抽出するのに適した層だと考えられる.1次元畳み込みはテキストや音声など,データの形状が$\lr{\dimUpper \times \timeUpper}$となっている時に用いられる.

これに加えて,カーネルを二次元配列とすれば二次元畳み込み層,三次元配列とすれば三次元畳み込み層となる.二次元畳み込み層は画像など,データの形状が$\lr{\dimUpper \times \heightUpper \times \widthUpper}$となっている場合に用いられる.ここで,$\heightUpper$は高さ,$\widthUpper$を表す.三次元畳み込み層は動画など,データの形状が$\lr{\dimUpper \times \heightUpper \times \widthUpper \times \timeUpper}$となっている場合に用いられる.

畳み込み層における主要なパラメータは3つある.1つ目はカーネルサイズであり,これによって考慮できる入力特徴量の範囲が定まる.2つ目はストライドであり,これによってカーネルのシフト幅を設定できる.3つ目はダイレーションであり,これは畳み込み演算において計算対象となる入力特徴量の間隔を表す.ダイレーションを大きくすることで,カーネルサイズが同じでも考慮できる入力特徴量の範囲を広げることが可能である.また,出力系列長$\timeUpper_{\text{out}}$を入力系列長$\timeUpper_{\text{in}}$の整数倍に保つには,上記のパラメータに対して適切なパディング長を指定する必要がある.図\ref{sec3:fig:conv_variations}に,ある入出力次元間における1次元畳み込み層の処理を示す.赤が入出力,青がカーネルを表している.図\ref{sec3:fig:conv1}は,カーネルサイズを3,ストライドとダイレーションを1とした場合であり,入力の両端に1ずつゼロパディングすることで,入出力の系列長が変わらないことを表している.図\ref{sec3:fig:conv2}は,カーネルサイズを5,ストライドとダイレーションを1とした場合であり,カーネルサイズが大きくなることで,出力1フレームを計算する際に考慮される入力の範囲が図\ref{sec3:fig:conv1}よりも広がることを表した.図\ref{sec3:fig:conv3}はカーネルサイズが3,ストライドが2,ダイレーションが1の場合であり,ストライドを2にしたことによって出力の系列長が入力よりも短くなることを表している.ここではパディングを1とすることで,入力に対して出力の系列長を$1/2$にできることを表した.図\ref{sec3:fig:conv4}はカーネルサイズが3,ストライドが1,ダイレーションが2の場合であり,ダイレーションを2とすることで,カーネルに対して入力が1フレーム飛ばしで計算されることを示した.

\begin{figure}[tb]
    \centering
    \begin{subfigure}[b]{0.48\textwidth}
        \centering
        \includegraphics[height=4cm]{./figure/sec3/conv1.drawio.png}
        \caption{$\lr{\kernelSizeUpper, \strideUpper, \dilationUpper} = \lr{3, 1, 1}$}
        \label{sec3:fig:conv1}
    \end{subfigure}
    \begin{subfigure}[b]{0.48\textwidth}
        \centering
        \includegraphics[height=4cm]{./figure/sec3/conv2.drawio.png}
        \caption{$\lr{\kernelSizeUpper, \strideUpper, \dilationUpper} = \lr{5, 1, 1}$}
        \label{sec3:fig:conv2}
    \end{subfigure}

    \vspace{0.5cm}

    \begin{subfigure}[b]{0.48\textwidth}
        \centering
        \includegraphics[height=4cm]{./figure/sec3/conv3.drawio.png}
        \caption{$\lr{\kernelSizeUpper, \strideUpper, \dilationUpper} = \lr{3, 2, 1}$}
        \label{sec3:fig:conv3}
    \end{subfigure}
    \begin{subfigure}[b]{0.48\textwidth}
        \centering
        \includegraphics[height=4cm]{./figure/sec3/conv4.drawio.png}
        \caption{$\lr{\kernelSizeUpper, \strideUpper, \dilationUpper} = \lr{3, 1, 2}$}
        \label{sec3:fig:conv4}
    \end{subfigure}
    \caption{ある入出力次元間における1次元畳み込み層の処理.$\kernelSizeUpper$はカーネルサイズ,$\strideUpper$はストライド,$\dilationUpper$はダイレーションを表し,図中の0はパディング部を表す.}
    \label{sec3:fig:conv_variations}
\end{figure}

\subsubsection{転置畳み込み層}
転置畳み込み層は,畳み込み層の逆演算に対応する層であり,主に入力のアップサンプリングに使用される.図\ref{sec3:fig:tconv_variations}に,ある入出力次元間における1次元転置畳み込み層の処理を示す.赤が入出力,青がカーネルを表している.1次元転置畳み込み層では,$\timeLower$番目の入力とカーネルの積を計算し,その結果を$\timeLower$番目から$\timeLower + \kernelSizeUpper$番目までの出力とする.ここで$\kernelSizeUpper$はカーネルサイズである.また,複数の入力から計算された出力がオーバーラップする場合,これらは加算される.図\ref{sec3:fig:tconv1}は,カーネルサイズを4,ストライドを1とした場合の様子である.アップサンプリングを行いたい場合は,ストライドを2以上とすれば良い.図\ref{sec3:fig:tconv2}にカーネルサイズを4,ストライドを2とした場合を示す.この時,入力系列長が4であるのに対して,出力系列長が10まで拡大されていることがわかる.ここで,出力系列長を入力系列長の整数倍に保つためには,出力の両端の削除数を適切に設定する必要がある.上述の例では両端の削除数を1とすることで,出力系列長を入力系列長の2倍である8に調整できる.

\begin{figure}[tb]
    \centering
    \begin{subfigure}[b]{0.48\textwidth}
        \centering
        \includegraphics[height=4cm]{./figure/sec3/tconv1.drawio.png}
        \caption{$\lr{\kernelSizeUpper, \strideUpper} = \lr{4, 1}$}
        \label{sec3:fig:tconv1}
    \end{subfigure}
    \begin{subfigure}[b]{0.48\textwidth}
        \centering
        \includegraphics[height=4cm]{./figure/sec3/tconv2.drawio.png}
        \caption{$\lr{\kernelSizeUpper, \strideUpper} = \lr{4, 2}$}
        \label{sec3:fig:tconv2}
    \end{subfigure}
    \caption{ある入出力次元間における1次元転置畳み込み層の処理.$\kernelSizeUpper$はカーネルサイズ,$\strideUpper$はストライドを表す.}
    \label{sec3:fig:tconv_variations}
\end{figure}

\subsubsection{活性化関数}
活性化関数は,ニューラルネットワークの出力に非線形性を与えるための関数である.これにより,DNNは単純な線形変換だけでは表現できない複雑な入出力の関係を学習可能になる.以下,活性化関数への入力を$\inputLower \in \realSet$として,代表的なものを6つ述べる.また,本節で取り上げる活性化関数とその1階導関数のグラフを図\ref{sec3:fig:activations_and_their_prime}に示す.

1つ目は,シグモイド関数である.シグモイド関数は
\begin{equation}
    \sigmoid\lr{\inputLower} = \frac{1}{1 + \exp\lr{-\inputLower}}
\end{equation}
で与えられ,その1階導関数は
\begin{equation}
    \dv{\sigmoid\lr{\inputLower}}{x} = \frac{\exp\lr{-\inputLower}}{\lr{1 + \exp\lr{-\inputLower}}^{2}}
\end{equation}
となる.図\ref{sec3:fig:activations_prime}より,シグモイド関数の1階導関数の最大値は$\inputLower=0$における0.25であり,$\lrAbs{\inputLower}$が大きくなるのに伴って,1階導関数の値は小さくなることがわかる.DNNの各重みは,損失関数の勾配を利用することで更新されるから,シグモイド関数以前の層の重みにおける勾配は,シグモイド関数の1階導関数の値が乗算された結果となる.前述したように,シグモイド関数は1階導関数の値が小さくなりがちであるから,それ以前の層における勾配も小さくなり,重みの更新が進みづらくなる可能性がある.この問題を,勾配消失と呼ぶ.

2つ目は,$\tanh$関数である.$\tanh$は
\begin{equation}
    \tanh\lr{\inputLower} = \frac{\exp\lr{\inputLower} - \exp\lr{-\inputLower}}{\exp\lr{\inputLower} + \exp\lr{-\inputLower}}
\end{equation}
で与えられ,その1階導関数は
\begin{align}
    \dv{\tanh\lr{\inputLower}}{x} & = \frac{4}{\lr{\exp\lr{\inputLower} + \exp\lr{-\inputLower}}^{2}} \\
                                  & = \frac{1}{\cosh\lr{\inputLower}^{2}}
\end{align}
となる.$\tanh$の値域は$\lrClosedInterval{-1}{1}$となっており,図\ref{sec3:fig:activations_prime}より$\lrAbs{\inputLower}$が小さいところではシグモイド関数より1階導関数の値が大きくなっていることがわかる.しかし,$\lrAbs{\inputLower}$が大きくなればシグモイド関数と同様に1階導関数の値が小さく,勾配消失のリスクを抱えていることがわかる.

3つ目は,$\relu$である.$\relu$は
\begin{equation}
    \relu\lr{\inputLower} = \max \lr{0, \inputLower}
\end{equation}
で与えられ,その1階導関数は
\begin{equation}
    \dv{\relu\lr{\inputLower}}{x} =
    \begin{cases}
        1 & \text{if $\inputLower > 0$}  \\
        0 & \text{if $\inputLower <= 0$}
    \end{cases}
\end{equation}
となる.ここで,$\relu$は本来$x = 0$で微分不可能であるが,便宜上$\dv*{\relu\lr{0}}{x} = 0$とした.$\relu$は入力が0以上であれば恒等写像として振る舞うが,0未満であれば0に写す.1階導関数は0あるいは1のみを取り,特に入力が正の値であれば常に1となることから,シグモイド関数や$\tanh$よりも勾配消失が起こりづらい.$\relu$は現在,標準的な活性化関数として広く用いられている.しかし,$\relu$への入力が0未満の値を取るとき,$\relu$入力についての出力の勾配は0になるから,$\relu$以前の層の重みが更新されず,学習が遅くなる可能性がある.

3つ目は,$\leakyRelu$\cite{maas2013rectifier}である.$\leakyRelu$は
\begin{equation}
    \leakyRelu\lr{\inputLower} =
    \begin{cases}
        \inputLower  & \text{if $\inputLower > 0$}  \\
        a\inputLower & \text{if $\inputLower <= 0$}
    \end{cases}
\end{equation}
で与えられ,その1階導関数は
\begin{equation}
    \dv{\leakyRelu\lr{\inputLower}}{x} =
    \begin{cases}
        1 & \text{if $\inputLower > 0$}  \\
        a & \text{if $\inputLower <= 0$}
    \end{cases}
\end{equation}
となる.ここで,$\leakyRelu$は本来$x = 0$で微分不可能であるが,便宜上$\dv*{\leakyRelu\lr{0}}{x} = a$とした.ReLUと比較すると,0未満の入力に対しても0でない値を出力し,1階導関数も0にならない点が異なっている.これにより,重みの更新が進まなくなるReLUの課題を解消した.

5つ目は,$\prelu$\cite{he2015delving}である.これは,$\leakyRelu$と似た活性化関数であるが,$\leakyRelu$のパラメータ$a$を学習可能にすることで,その他の層と合わせて最適化が可能となったことが特徴である.

6つ目は,$\gelu$\cite{hendrycks2016gaussian}である.$\gelu$は
\begin{equation}
    \gelu\lr{\inputLower} = \inputLower \Phi\lr{\inputLower}
\end{equation}
で与えられる.ここで,
\begin{equation}
    \Phi\lr{\inputLower} = P\lr{X \le \inputLower}, ~ X \sim \mathcal{N} \lr{0, 1}
\end{equation}
である.$\gelu$の1階導関数は,
\begin{equation}
    \dv{\gelu\lr{\inputLower}}{x} = \Phi\lr{\inputLower} + \frac{\inputLower}{\sqrt{2\pi}}\exp\lr{-\frac{\inputLower^{2}}{2}}
\end{equation}
となる.$\gelu$は,$\relu$が入力に対して0あるいは1を確定的にかける活性化関数と捉えた上で,これを入力に依存した確率的な挙動に変更したものである.
% 実際,$\binaryMaskLower \sim \text{Bernoulli}\lr{\Phi\lr{\inputLower}}$とすると,
% \begin{align}
%     \gelu\lr{\inputLower} & = \inputLower \Phi\lr{\inputLower}                                                             \\
%                           & = 1 \inputLower \cdot \Phi\lr{\inputLower} + 0 \inputLower \cdot \lr{1 - \Phi\lr{\inputLower}} \\
%                           & = \expectation{mx}
% \end{align}
% となり,$\gelu$の出力は確率的なバイナリマスク$\binaryMaskLower$を入力$\inputLower$にかけた,$mx$の期待値に等しいことがわかる.

\begin{figure}[tb]
    \centering
    \begin{subfigure}[b]{1.0\textwidth}
        \centering
        \includegraphics[width=160  mm]{./figure/sec3/activations.png}
        \caption{活性化関数}
        \label{sec3:fig:activations}
    \end{subfigure}
    \begin{subfigure}[b]{1.0\textwidth}
        \centering
        \includegraphics[width=160  mm]{./figure/sec3/activations_prime.png}
        \caption{活性化関数の1階導関数}
        \label{sec3:fig:activations_prime}
    \end{subfigure}
    \caption{活性化関数の例}
    \label{sec3:fig:activations_and_their_prime}
\end{figure}

\subsubsection{再帰型ニューラルネットワーク}
再帰型ニューラルネットワーク(Recurrent Neural Network; RNN)は,自身の過去の出力を保持し,それをループさせる再帰的な構造を持ったネットワークである.

近年よく用いられるRNNとして,長・短期記憶(Long Short-Time Memory; LSTM)\cite{hochreiter1997long}がある.LSTMは入力ゲート,忘却ゲート,出力ゲートの3つを持ち,これらゲートによってネットワーク内部の情報の取捨選択を行うことで,長い系列データからの学習を可能にした.LSTMのネットワーク内部で行われる計算を以下に示す.
\begin{gather}
    \bm{f}_{\timeLower} = \sigmoid\lr{\fcLayerValue{\bm{\inputLower}_{\timeLower}}{\bm{\weightAndBias}_{f, x}} + \fcLayerValue{\bm{h}_{\timeLower-1}}{\bm{\weightAndBias}_{f, h}}} \\
    \bm{i}_{\timeLower} = \sigmoid\lr{\fcLayerValue{\bm{\inputLower}_{\timeLower}}{\bm{\weightAndBias}_{i, \inputLower}} + \fcLayerValue{\bm{h}_{\timeLower-1}}{\bm{\weightAndBias}_{i, h}}} \\
    \tilde{\bm{c}}_{\timeLower} = \tanh\lr{\fcLayerValue{\bm{\inputLower}_{\timeLower}}{\bm{\weightAndBias}_{\tilde{c}, \inputLower}} + \fcLayerValue{\bm{h}_{\timeLower-1}}{\bm{\weightAndBias}_{\tilde{c}, h}}} \\
    \bm{o}_{\timeLower} = \sigmoid\lr{\fcLayerValue{\bm{\inputLower}_{\timeLower}}{\bm{\weightAndBias}_{o, \inputLower}} + \fcLayerValue{\bm{h}_{\timeLower-1}}{\bm{\weightAndBias}_{o, h}}} \\
    \bm{c}_{\timeLower} = \bm{f}_{\timeLower} \elemMul \bm{c}_{\timeLower-1} + \bm{i}_{\timeLower} \elemMul \tilde{\bm{c}}_{\timeLower} \\
    \bm{h}_{\timeLower} = \bm{o}_{\timeLower} \elemMul \tanh\lr{\bm{c}_{\timeLower}}
\end{gather}
ここで,$\bm{\inputLower}_{\timeLower} \in \realSet^{\dimUpper_{\text{in}}}$は時刻$\timeLower$の入力,$\bm{f}_{\timeLower} \in \lrsq{0, 1}^{\dimUpper_{\text{out}}}$は忘却ゲートの出力,$\bm{i}_{\timeLower} \in \lrsq{0, 1}^{\dimUpper_{\text{out}}}$は入力ゲートの出力,$\bm{c}_{\timeLower} \in \lrsq{-1, 1}^{\dimUpper_{\text{out}}}$は時刻$\timeLower$におけるセルの状態,$\bm{o}_{\timeLower} \in \lrsq{0, 1}^{\dimUpper_{\text{out}}}$は出力ゲートの出力,$\bm{h}_{\timeLower} \in \lrsq{-1, 1}^{\dimUpper_{\text{out}}}$は時刻$\timeLower$における隠れ状態,$\dimUpper_{\text{in}}, \dimUpper_{\text{out}}$は特徴量の次元を表す.また,$\concat{\cdot, \ldots, \cdot}$は入力された特徴量の次元方向の結合,$\elemMul$は要素積を表す.忘却ゲート出力$\bm{f}_{\timeLower}$が前時刻のセル状態$\bm{c}_{\timeLower - 1}$に含まれる情報の選択,入力ゲート出力$\bm{i}_{\timeLower}$が新たな入力$\tilde{\bm{c}}_{\timeLower}$に含まれる情報の選択に用いられ,$\bm{c}_{\timeLower}$が決まる.その後,出力ゲート出力$\bm{o}_{\timeLower}$が$\bm{c}_{\timeLower}$に含まれる情報の選択に用いられ,$\bm{h}_{\timeLower}$が決まる.

また,LSTMが3つのゲートを必要とするのに対し,ゲートを2つに減らすことでネットワークの軽量化を図ったのがゲート付き回帰型ユニット(Gated Recurrent Unit; GRU)\cite{cho2014learning}である.GRUはリセットゲートと更新ゲートの2つを用いて隠れ状態を更新する.GRUのネットワーク内部で行われる計算を以下に示す.
\begin{gather}
    \bm{z}_{\timeLower} = \sigmoid\lr{\fcLayerValue{\bm{\inputLower}_{\timeLower}}{\bm{\weightAndBias}_{z, \inputLower}} + \fcLayerValue{\bm{h}_{\timeLower-1}}{\bm{\weightAndBias}_{z, h}}} \\
    \bm{r}_{\timeLower} = \sigmoid\lr{\fcLayerValue{\bm{\inputLower}_{\timeLower}}{\bm{\weightAndBias}_{r, \inputLower}} + \fcLayerValue{\bm{h}_{\timeLower-1}}{\bm{\weightAndBias}_{r, h}}} \\
    \tilde{\bm{h}}_{\timeLower} = \tanh\lr{\fcLayerValue{\bm{\inputLower}_{\timeLower}}{\bm{\weightAndBias}_{\tilde{h}, \inputLower}} + \bm{r}_{\timeLower} \elemMul \fcLayerValue{\bm{h}_{\timeLower-1}}{\bm{\weightAndBias}_{\tilde{h}, h}}} \\
    \bm{h}_{\timeLower} = \lr{1 - \bm{z}_{\timeLower}} \elemMul \bm{h}_{\timeLower-1} + \bm{z}_{\timeLower} \elemMul \tilde{\bm{h}}_{\timeLower}
\end{gather}
ここで,$\bm{x}_{\timeLower} \in \realSet^{\dimUpper_{\text{in}}}$が時刻$\timeLower$における入力,$\bm{z}_{\timeLower} \in \lrsq{0, 1}^{\dimUpper_{\text{out}}}$が更新ゲートの出力,$\bm{r}_{\timeLower} \in \lrsq{0, 1}^{\dimUpper_{\text{out}}}$がリセットゲートの出力,$\bm{h}_{\timeLower} \in \lrsq{-1, 1}^{\dimUpper_{\text{out}}}$が時刻$\timeLower$における隠れ状態を表す.更新ゲート出力$\bm{z}_{\timeLower}$が$\bm{h}_{\timeLower - 1}$と$\tilde{\bm{h}}_{\timeLower}$に含まれる情報の選択,リセットゲート出力$\bm{r}_{\timeLower}$が$\bm{h}_{\timeLower - 1}$に含まれる情報の選択に用いられる.

\subsubsection{正規化層}
DNNの学習過程では学習の進行に伴って重みが変化するため,その度に各層への入力の分布が変わってしまう.これは内部共変量シフトと呼ばれ,ネットワークの学習を不安定にする原因となる.これに対し,バッチ正規化(Batch Normalization)\cite{ioffe2015batch}が有効である.バッチ正規化は,ミニバッチ内における入力特徴量の期待値と分散を次元ごとに計算し,これらを用いて入力特徴量を次元ごとに標準化するものである.ここで,バッチサイズを$\numUpper$,バッチ正規化への$D$次元の入力特徴量を$\bm{\inputLower}_{\numLower} \in \realSet^{\dimUpper} ~ \lr{\numLower = 1, \ldots, \numUpper}$,出力特徴量を$\bm{y}_{\numLower} \in \realSet^{\dimUpper} ~ \lr{\numLower = 1, \ldots, \numUpper}$とする.このとき,各$\numLower$に対し入力特徴量$\bm{\inputLower}_{\numLower}$の$\dimLower$次元目の成分を$\inputLower_{\numLower, \dimLower}$,出力特徴量$\bm{\outputLower}_{\numLower}$の$\dimLower$次元目の成分を$\outputLower_{\numLower, \dimLower}$とすると,$\outputLower_{\numLower, \dimLower}$は
\begin{align}
    \mean^{B}_{\dimLower}                      & = \frac{1}{\numUpper} \sum_{\numLower = 1}^{\numUpper} \inputLower_{\numLower, \dimLower}                                  \\
    \lr{\std^{B}_{\dimLower}}^{2}              & = \frac{1}{\numUpper} \sum_{\numLower = 1}^{\numUpper} \lr{\inputLower_{\numLower, \dimLower} - \mean^{B}_{\dimLower}}^{2} \\
    \tilde{\inputLower}_{\numLower, \dimLower} & = \frac{\inputLower_{\numLower, \dimLower} - \mean^{B}_{\dimLower}}{\sqrt{\lr{\std^{B}_{\dimLower}}^{2} + \epsilon}}       \\
    \outputLower_{\numLower, \dimLower}        & = \normScale_{\dimLower} \tilde{\inputLower}_{\numLower, \dimLower} +  \normShift_{\dimLower}
\end{align}
で与えられる.ここで,$\normScale_{\dimLower}, \normShift_{\dimLower}$は学習可能なスカラーであり,$\epsilon$はゼロ割を避けるためのスカラーである.バッチ正規化では$\normScale_{\dimLower}, \normShift_{\dimLower}$によって表現力を向上させており,実際
\begin{align}
    \normScale_{\dimLower} & = \sqrt{\lr{\std^{B}_{\dimLower}}^{2} + \epsilon} \\
    \beta_{\dimLower}      & = \mean^{B}_{\dimLower}
\end{align}
とすれば,標準化前の入力を再び得ることが可能である.学習時は,サンプルの標準化に用いる統計量とは別に,期待値の移動平均と不偏分散の移動平均を計算しておく.推論時は学習終了時に得られたこれら移動平均の値を用いるため,入力サンプルによらない挙動になる.

バッチ正規化はDNNの学習の安定化に貢献する一方,ミニバッチ全体における統計量を利用するため,バッチサイズが小さい場合はデータの分布を安定させることが難しくなる.また,テキストや音声といった系列長を持つデータを扱う場合,ミニバッチを構成するためにはゼロパディングによって系列長を揃える必要がある.この時,RNNの各ステップの出力に対しバッチ正規化を適用すると,ゼロパディングによって人為的に系列量を揃えているから,統計量が実際のデータの分布からかけ離れたものになる可能性がある.これら課題に対し,ミニバッチ内の各サンプルごとに期待値と分散を求めて標準化する,レイヤー正規化(Layer Normalization)\cite{ba2016layer}がある.バッチ正規化のときと同様の表記を用いると,$\outputLower_{\numLower, \dimLower}$は
\begin{align}
    \mean^{L}_{\numLower}                      & = \frac{1}{\dimUpper} \sum_{\dimLower = 1}^{\dimUpper} \inputLower_{\numLower, \dimLower}                                  \\
    \lr{\std^{L}_{\numLower}}^{2}              & = \frac{1}{\dimUpper} \sum_{\dimLower = 1}^{\dimUpper} \lr{\inputLower_{\numLower, \dimLower} - \mean^{L}_{\numLower}}^{2} \\
    \tilde{\inputLower}_{\numLower, \dimLower} & = \frac{\inputLower_{\numLower, \dimLower} - \mean^{L}_{\numLower}}{\sqrt{\lr{\std^{L}_{\numLower}}^{2} + \epsilon}}       \\
    \outputLower_{\numLower, \dimLower}        & = \normScale_{\dimLower} \tilde{\inputLower}_{\numLower, \dimLower} +  \normShift_{\dimLower}
\end{align}
で与えられる.

上述したバッチ正規化およびレイヤー正規化は,特徴量を標準化することで学習を安定させる手法であった.一方,DNN内のある層の重みを再パラメータ化することで学習を安定させる手法として,重み正規化(Weight Normalization)\cite{salimans2016weight}がある.これは,ある層の重みベクトル$\bm{\weightLower}$を,
\begin{equation}
    \bm{\weightLower} = \frac{\bm{v}}{\lrTwoNorm{\bm{v}}} g
\end{equation}
のように単位ベクトル$\bm{v} / \lrTwoNorm{\bm{v}}$(ベクトルの向き)とスカラー$g$(ベクトルの長さ)に再パラメータ化するものである.学習時は重みの更新を$\bm{v}$と$g$で別々に行う.重み正規化は,バッチ正規化やレイヤー正規化と同様に学習の安定化に役立つが,計算に入力特徴量の系列長が依存しない.そのため,例えば音声波形など系列長が非常に長くなりがちなデータを扱う場合,計算コストを下げながら同様の効果を狙える手段だと考えられる.

\subsubsection{Transformer}
Transformer\cite{vaswani2017attention}は,自己注意機構(Self-Attention)を用いて,入力系列全体に渡る依存関係を捉えることができるニューラルネットワークである.特に,再帰的な計算を必要とするRNNと比較して,Transformerは並列計算のみ行うため,GPUによる計算の高速化が可能である.以下,入力特徴量を$\bm{\inputUpper} \in \realSet^{\timeUpper \times \dimUpper_{\text{model}}}$として,Transformerにおいて行われる計算を説明する.$T$は系列長,$\dimUpper_{\text{model}}$は特徴量の次元を表す.また,Transformer層の構造を図\ref{sec3:fig:transformer_layer}に示す.赤の平行四辺形が入出力,青の長方形が処理を表す.

\begin{figure}[bt]
    \centering
    \includegraphics[width=150mm]{./figure/sec3/transformer.drawio.png}
    \caption{Transformer層の構造(赤い平行四辺形:入出力,青い長方形:処理)}
    \label{sec3:fig:transformer_layer}
\end{figure}

まず,TransformerにおけるSelf-Attentionの計算の流れを述べる.ここでは,はじめにクエリ$\bm{Q} \in \realSet^{\timeUpper \times \dimUpper_{k}}$,キー$\bm{K} \in \realSet^{\timeUpper \times \dimUpper_{k}}$,バリュー$\bm{V} \in \realSet^{\timeUpper \times \dimUpper_{v}}$の計算を行う.これは,
\begin{align}
    \bm{Q} & = \fcLayerValue{\bm{\inputUpper}}{\bm{\weightAndBias}_{Q}} \\
    \bm{K} & = \fcLayerValue{\bm{\inputUpper}}{\bm{\weightAndBias}_{K}} \\
    \bm{V} & = \fcLayerValue{\bm{\inputUpper}}{\bm{\weightAndBias}_{V}}
\end{align}
で与えられる.次に,クエリとキーを元にアテンション重みを求め,バリューに対する行列積を計算する.これは,
\begin{equation}
    \text{SA}\lr{\bm{Q}, \bm{K}, \bm{V}} = \text{softmax}\lr{\frac{\bm{Q}\bm{K}^\top}{\sqrt{\dimUpper_{k}}}} \bm{V}
\end{equation}
で与えられる.softmax関数は行方向に適用されるため,$\text{softmax}\lr{\bm{Q}\bm{K}^\top / \sqrt{\dimUpper_{k}}}$の各行ベクトルが,各クエリ$\bm{q}_{t} \in \realSet^{\dimUpper_{k}}$からキー$\bm{k}_{t} \in \realSet^{\dimUpper_{k}} ~ \lr{t = 1, \ldots, T}$に対する注意度になっている.

ここまでがSelf-Attentionの計算であったが,TransformerではSelf-Attentionを複数のヘッドで並列に計算し,各ヘッドの出力を結合して最終出力を得るMulti-Head Attentionが採用されている.これは,ヘッド数を$\numHeadUpper$とし,各ヘッド$\numHeadLower$におけるクエリ$\bm{Q}^{\numHeadLower} \in \realSet^{\timeUpper \times \frac{\dimUpper_{k}}{\numHeadUpper}}$,キー$\bm{K}^{\numHeadLower} \in \realSet^{\timeUpper \times \frac{\dimUpper_{k}}{\numHeadUpper}}$,バリュー$\bm{V}^{\numHeadLower} \in \realSet^{\timeUpper \times \frac{\dimUpper_{v}}{\numHeadUpper}}$によって計算された$\text{SA}\lr{\bm{Q}^{\numHeadLower}, \bm{K}^{\numHeadLower}, \bm{V}^{\numHeadLower}}$を用いて,
\begin{equation}
    \text{MHA}\lr{\bm{Q}, \bm{K}, \bm{V}} = \fcLayerValue{\concat{\text{SA}\lr{\bm{Q}^{\numHeadLower}, \bm{K}^{\numHeadLower}, \bm{V}^{\numHeadLower}}, \ldots, \text{SA}\lr{\bm{Q}^{\numHeadLower}, \bm{K}^{\numHeadLower}, \bm{V}^{\numHeadLower}}}}{\bm{\weightAndBias}_{\text{MHA}}}
\end{equation}
で与えられる.Multi-Head Attention後は,残差結合とレイヤー正規化を適用する.すなわち,この出力$\bm{\outputUpper} \in \realSet^{\timeUpper \times \dimUpper_{\text{model}}}$は,
\begin{equation}
    \bm{\outputUpper} = \text{LayerNorm}\lr{\text{MHA}\lr{\bm{Q}, \bm{K}, \bm{V}} + \bm{\inputUpper}}
\end{equation}
で与えられる.その後,全結合層を通し,再度残差結合とレイヤー正規化を適用することでTransformer層最終出力を得る.すなわち,
\begin{equation}
    \bm{\outputUpper} = \text{LayerNorm}\lr{\fcLayerValue{\relu\lr{\fcLayerValue{\bm{\outputUpper}}{\bm{\weightAndBias}_{1}}}}{\bm{\weightAndBias}_{2}} + \bm{\outputUpper}}
\end{equation}
となる.Transformer全体は,Transformer層を多層積み重ねて構成される.

最後に,TransformerではRNNと違い,並列計算によって系列全体を一度に処理することが可能であるが,それと引き換えに入力の順序情報を考慮することができなくなる.これに対し,TransformerではPositional Encodingによって入力に位置情報を与える.Positional Encodingは$\sin$と$\cos$に基づいて,
\begin{equation}
    \text{PositionalEncoding}\lr{\timeLower, \dimLower} =
    \begin{cases}
        \sin \lr{\frac{\timeLower}{10000^{2\dimLower / \dimUpper_{\text{model}}}}} & \text{if $d \bmod 2 = 0$} \\
        \cos \lr{\frac{\timeLower}{10000^{2\dimLower / \dimUpper_{\text{model}}}}} & \text{if $d \bmod 2 = 1$}
    \end{cases}
\end{equation}
で与えられる.

\subsection{学習方法}
本節において,$\bm{\weightAndBias}$はDNNの全ての重みとバイアスをまとめて表す変数とする.また,文章中では簡潔さを優先し,特別な理由がない限りは重みと呼ぶ.

\subsubsection{損失関数}
損失関数は,DNNによって予測された結果と正解値との間の誤差を求める関数のことであり,扱う問題によって様々である.例えば,回帰問題において用いられる関数の1つに,MAE(Mean Absolute Error)Lossがある.$D$次元の予測対象を$\bm{\outputLower} \in \realSet^{\dimUpper}$,DNNによる予測結果を$\hat{\bm{\outputLower}} \in \realSet^{\dimUpper}$とすると,MAE Lossは
\begin{equation}
    \lossFuncUpper_{\text{MAE}}\lr{\bm{\outputLower}, \hat{\bm{\outputLower}}} = \frac{1}{\dimUpper} \sum_{\dimLower = 1}^{\dimUpper}  \lrAbs{\outputLower_{\dimLower} - \hat{\outputLower}_{\dimLower}}
\end{equation}
で与えられる.また,予測対象が系列長$T$を持った行列$\bm{\outputUpper} \in \realSet^{\timeUpper \times \dimUpper}$の場合,DNNによる予測結果を$\hat{\bm{\outputUpper}} \in \realSet^{\timeUpper \times \dimUpper}$とすると,MAE Lossは
\begin{equation}
    \lossFuncUpper_{\text{MAE}}\lr{\bm{\outputUpper}, \hat{\bm{\outputUpper}}} = \frac{1}{\timeUpper \dimUpper} \sum_{\timeLower = 1}^{\timeUpper} \sum_{\dimLower = 1}^{\dimUpper} \lrAbs{\outputLower_{\timeLower, \dimLower} - \hat{\outputLower}_{\timeLower, \dimLower}}
\end{equation}
で与えられる.

一方,分類問題において用いられる関数の1つに,Cross Entropy Lossがある.$\classUpper$クラス分類の問題について,予測対象を$\bm{\outputLower} \in \lrClosedInterval{0}{1}^{\classUpper}$,DNNによる予測値を$\hat{\bm{\outputLower}} \in \realSet^{\classUpper}$とすると,Cross Entropy Lossは
\begin{equation}
    \lossFuncUpper_{\text{CE}}\lr{\bm{\outputLower}, \hat{\bm{\outputLower}}} = - \sum_{\classLower = 1}^{\classUpper} y_{\classLower} \log \lr{\frac{\exp\lr{\hat{\outputLower}_{c}}}{\sum_{\indexLower = 1}^{C} \exp\lr{\hat{\outputLower}_{c}}}}
\end{equation}
で与えられる.ここで,$\bm{\outputLower}$はクラスに対する確率分布であり,
\begin{equation}
    \sum_{\classLower = 1}^{\classUpper} y_{\classLower} = 1
\end{equation}
を満たす.実際は,正解となるクラスのみを1,それ以外を0としたOne-hotベクトルとされることが多い.また,予測値$\hat{\bm{\outputLower}}$は各クラスに対するスコアを表す値であり,ロジットと呼ばれる.予測対象が行列$\bm{\outputUpper} \in \lrClosedInterval{0}{1}^{\timeUpper \times \classUpper}$の場合,ロジットを$\hat{\bm{\outputUpper}} \in \realSet^{\timeUpper \times \dimUpper}$とすると,Cross Entropy Lossは
\begin{equation}
    \lossFuncUpper_{\text{CE}}\lr{\bm{\outputUpper}, \hat{\bm{\outputUpper}}} = - \frac{1}{\timeUpper} \sum_{\timeLower = 1}^{\timeUpper} \sum_{\classLower = 1}^{\classUpper} y_{\timeLower, \classLower} \log \lr{\frac{\exp\lr{\hat{\outputLower}_{\timeLower, \classLower}}}{\sum_{\indexLower = 1}^{C} \exp\lr{\hat{\outputLower}_{\timeLower, \indexLower}}}}
\end{equation}
で与えられる.

\subsubsection{勾配降下法}
\label{sec3:sec:gradient_descent}
勾配降下法は,損失関数の重みについての勾配を利用して,損失関数の値を最小化するようにDNNを最適化するアルゴリズムである.ここで,学習データセットを$\datasetTrain = \lrc{\lr{\bm{\inputLower}_{\numLower}, \bm{\outputLower}_{\numLower}}}_{\numLower = 1}^{\numUpper}$とする.各$\numLower$に対し,$\bm{\inputLower}_{\numLower} \in \realSet^{\dimUpper_{\text{in}}}$はDNNへの入力, $\bm{\outputLower}_{\numLower} \in \realSet^{\dimUpper_{\text{out}}}$は予測対象を表す.$\dimUpper_{\text{in}}, \dimUpper_{\text{out}}$は入出力の次元である.DNNは$f$とし,予測値を$\hat{\bm{\outputLower}}_{\numLower} = f\lr{\bm{\inputLower}_{\numLower}; \bm{\weightAndBias}}$とする.損失関数は$\lossFuncUpper$とし,$\mathcal{\lossFuncUpper}$を
\begin{equation}
    \mathcal{\lossFuncUpper}\lr{\bm{\weightAndBias}; \mathcal{\indexUpper}} = \frac{1}{\card{\mathcal{\indexUpper}}} \sum_{\indexLower \in \mathcal{\indexUpper}} \lossFuncUpper\lr{\bm{\outputLower}_{\indexLower}, \hat{\bm{\outputLower}}_{\indexLower}} = \frac{1}{\card{\mathcal{\indexUpper}}} \sum_{\indexLower \in \mathcal{\indexUpper}} \lossFuncUpper\lr{\bm{\outputLower}_{\indexLower}, f\lr{\bm{\inputLower}_{\indexLower}; \bm{\weightAndBias}}}
\end{equation}
で定義する.ここで,$\mathcal{\indexUpper} \subset \lrc{1, \ldots, N}$は各サンプルに対するインデックスの部分集合,$\card{\cdot}$は集合の濃度を表す.$\mathcal{\lossFuncUpper}$は学習データ$\lrc{\lr{\bm{\inputLower}_{\indexLower}, \bm{\outputLower}_{\indexLower}}}_{\indexLower \in \indexUpper} \subset \mathcal{D}_{\text{train}}$に対する損失を,DNNの重み$\bm{\weightAndBias}$の関数として扱うために導入した.上の表記を用いると,DNNの最適化問題は
\begin{equation}
    \min_{\bm{\weightAndBias}} \mathcal{\lossFuncUpper}\lr{\bm{\weightAndBias}; \lrc{1, \ldots, N}}
\end{equation}
と表される.この最適化問題に対し,勾配降下法による重み$\bm{\weightAndBias}$の更新は,
\begin{equation}
    \label{sec3:eq:normal_gradient_descent}
    \bm{\weightAndBias}_{\iter} = \bm{\weightAndBias}_{\iter - 1} - \learningRate \nabla_{\bm{\weightAndBias}} \mathcal{\lossFuncUpper}\lr{\bm{\weightAndBias}_{\iter - 1}; \mathcal{I}_{\iter}}
\end{equation}
で与えられる.ここで,$\iter$は学習におけるイテレーション,$\learningRate$は学習率を表す.

勾配降下法には,3種類の方法がある\cite{zhang2019gradient}.1つ目は,バッチ勾配降下法である.これは,
\begin{equation}
    \bm{\weightAndBias}_{\iter} = \bm{\weightAndBias}_{\iter - 1} - \learningRate \nabla_{\bm{\weightAndBias}} \mathcal{\lossFuncUpper}\lr{\bm{\weightAndBias}_{\iter - 1}; \lrc{1, \ldots, N}}
\end{equation}
で与えられる.すなわち,各イテレーションで学習データ全てを用いる方法である.各サンプルのノイズの影響が低減されることで安定した学習が期待できるが,計算コストが高い.2つ目は,確率的勾配降下法である.これは,ランダムに選択された$n_{\iter} \in \lrc{1, \ldots, N}$に対し,
\begin{equation}
    \bm{\weightAndBias}_{\iter} = \bm{\weightAndBias}_{\iter - 1} - \learningRate \nabla_{\bm{\weightAndBias}} \mathcal{\lossFuncUpper}\lr{\bm{\weightAndBias}_{\iter - 1}; \lrc{n_{\iter}}}
\end{equation}
で与えられる.すなわち,各イテレーションで単一サンプルのみを用いる方法である.計算コストが下がるが,各サンプルのノイズの影響が大きくなることで学習が不安定になる可能性がある.3つ目は,ミニバッチ勾配降下法である.これは,$1 < \card{\mathcal{I}_{\iter}} < N$を満たすランダムに選択された$\mathcal{I}_{\iter} \subsetneq \lrc{1, \ldots, N}$に対し,
\begin{equation}
    \bm{\weightAndBias}_{\iter} = \bm{\weightAndBias}_{\iter - 1} - \learningRate \nabla_{\bm{\weightAndBias}} \mathcal{\lossFuncUpper}\lr{\bm{\weightAndBias}_{\iter - 1}; \mathcal{I}_{\iter}}
\end{equation}
で与えられる.バッチ勾配降下法と確率的勾配降下法の間をとった方法であり,DNNの学習においては一般にミニバッチ勾配降下法が用いられる.ここで,ミニバッチに含まれるサンプルの数をバッチサイズと呼ぶ.

また,確率的勾配降下法やミニバッチ勾配降下法では,サンプルを学習データセットからランダムに非復元抽出する.ここで,毎回のサンプリングされた学習データに対する処理は1イテレーションとカウントし,学習データセットを一度全て使い切ることは1エポックとカウントする.実際には,データセットの総サンプル数$N$に対してバッチサイズを決定することで1エポックあたりの総イテレーション数は決まり,最大エポック数を設定して学習を回すこととなる.

\subsubsection{正則化}
DNNは大量のパラメータにより高い表現力を持つが,その分学習データに過剰に適合し,未知データに対する汎化性能が低いモデルとなる,過学習を引き起こす可能性がある.正則化は,このようなDNNの過学習を防ぐための手段である.以下,具体的な方法を3つ述べる.

1つ目は,L2正則化である.これは,\ref{sec3:sec:gradient_descent}節で定義した$\mathcal{\lossFuncUpper}\lr{\bm{\weightAndBias}; \mathcal{\indexUpper}}$を,
\begin{equation}
    \label{sec3:eq:l2_reg}
    \mathcal{\lossFuncUpper}\lr{\bm{\weightAndBias}; \mathcal{\indexUpper}} = \frac{1}{\card{\mathcal{\indexUpper}}} \sum_{\indexLower \in \mathcal{\indexUpper}} \lossFuncUpper\lr{\bm{\outputLower}_{\indexLower}, f\lr{\bm{\inputLower}_{\indexLower}; \bm{\weightAndBias}}} + \frac{\regConst}{2} \lrTwoNorm{\bm{\weightAndBias}}^{2}
\end{equation}
とすることで与えられる.ここで,$\regConst$は正則化の程度を調整するパラメータである.これより,L2正則化は損失関数の値に$\lrTwoNorm{\bm{\weightAndBias}}^{2}$を加算することで,重み$\bm{\weightAndBias}$のL2ノルムが過大になることを防ぐ方法だと言える.

2つ目は,Weight Decayである.これは,重みの更新式を
\begin{equation}
    \label{sec3:eq:weight_decay}
    \bm{\weightAndBias}_{\iter} = \bm{\weightAndBias}_{\iter - 1} - \learningRate \nabla_{\bm{\weightAndBias}} \mathcal{\lossFuncUpper}\lr{\bm{\weightAndBias}_{\iter - 1}; \mathcal{I}_{\iter}} - \regConst' \bm{\weightAndBias}
\end{equation}
とすることで与えられる.ここで,$\regConst'$は正則化の程度を調整するパラメータである.これより,Weight Decayは重み$\bm{\weightAndBias}$の絶対値が過大になることを防ぐ手法だと言える.

3つ目は,Dropout\cite{srivastava2014dropout}である.Dropoutは,学習時に特徴量の一部を0に落とす手法である.一方,推論時は恒等写像となり,学習時と挙動が変わる.Dropoutへの入力を$\bm{\inputLower} \in \realSet^{\dimUpper}$,特徴量を0に落とす確率を$p$とすると,学習時のDropout出力$\bm{\outputLower}_{\text{train}} \in \realSet^{\dimUpper}$および推論時のDropout出力$\bm{\outputLower}_{\text{infer}} \in \realSet^{\dimUpper}$は,
\begin{align}
    \bm{\outputLower}_{\text{train}} & = \frac{\bm{x} \elemMul \bm{m}}{1 - p} \label{sec3:eq:regularization_dropout_training_output} \\
    \bm{\outputLower}_{\text{infer}} & = \bm{x}
\end{align}
で与えられる.ここで,$D$は特徴量の次元,$\bm{m} \in \lrc{0, 1}^{\dimUpper}$は各成分$m_{\dimLower}$が確率$1 - p$で1,確率$p$で0をとる確率変数である.式\eqref{sec3:eq:regularization_dropout_training_output}より,学習時はDropout出力を$1 / \lr{1 - p}$倍していることがわかる.この理由は,学習時の出力の期待値と推論時の出力を一致させるためである.実際,確率変数$\bm{m}$の従う確率分布上で$\bm{\outputLower}_{\text{train}}$の期待値をとれば,
\begin{align}
    \expectation{\bm{\outputLower}_{\text{train}}} & = \expectation{\frac{\bm{x} \elemMul \bm{m}}{1 - p}}             \\
                                                   & = \frac{\bm{x}}{1 - p} \elemMul \expectation{\bm{m}}             \\
                                                   & = \frac{\bm{x}}{1 - p} \elemMul \lr{1 - p ~ \cdots ~ 1 - p}^\top \\
                                                   & = \bm{x} = \bm{\outputLower}_{\text{infer}}
\end{align}
となる.Dropoutは学習時に一部のニューロンを落とすことで,実質的に異なるネットワークを学習させていると考えることができる.よって,学習時の出力の期待値に推論時の出力を一致させることは,異なるネットワークから得られた出力の期待値をとり,アンサンブルモデルとして推論時の出力を得ていると解釈できる.

4つ目は,Early Stoppingである.Early Stoppingは,検証データに対する損失の増加を監視し,設定したエポック数だけ増加し続けた場合に学習を停止する手法である.これにより,学習データに対する過度なフィッティングを防止する.

\subsubsection{最適化手法}
\label{sec3:sec:optimizer}
\ref{sec3:sec:gradient_descent}節において,DNNの重みが勾配降下法によって最適化されることを述べた.ここで,通常の勾配降下法に代わり,近年よく用いられる最適化手法としてAdam\cite{kingma2014adam}がある.Adamの計算過程をアルゴリズム\ref{sec3:algo:adam}に示す.ここで,$\bm{g}_{\iter}$は勾配の一次モーメント,$\bm{m}_{\iter}$が勾配の一次モーメントの指数移動平均,$\bm{v}_{\iter}$が勾配の二次モーメントの指数移動平均である.$\hat{\bm{m}}_{\iter}$および$\hat{\bm{v}}_{\iter}$はそれぞれ,$\bm{m}_{\iter}$および$\bm{v}_{\iter}$の初期値が0であることに起因するバイアスを防ぐための計算を行った結果である.$\optimEmaConst_{1}, \optimEmaConst_{2}$は,指数移動平均の程度を調整するパラメータである.また,9行目のベクトルとベクトルの割り算は,各成分ごとに行われる.Adamでは,$\hat{\bm{m}}_{\iter}$を勾配の代わりに用いることで,ミニバッチごとのノイズに対する頑健な学習を可能にする.また,$\hat{\bm{v}}_{\iter}$の平方根で$\hat{\bm{m}}_{\iter}$を割ることによって,勾配の大きさに対するスケーリングを行っている.勾配が大きすぎる場合や小さすぎる場合を避けることで,勾配爆発や勾配消失を回避する効果があると考えられる.
\begin{algorithm}
    \caption{Adam}
    \label{sec3:algo:adam}
    \begin{algorithmic}[1]
        \State \textbf{Input:} $\learningRate$, $\optimEmaConst_{1}$, $\optimEmaConst_{2}$, $\regConst$, $\bm{\weightAndBias}_{0}$, $L\lr{\bm{\weightAndBias}}$
        \State $\bm{m}_{0} = 0$, $\bm{v}_{0} = 0$
        \For{$\iter = 1$ to \texttt{...}}
        \State $\bm{g}_{\iter} = \nabla_{\bm{\weightAndBias}} \mathcal{\lossFuncUpper}\lr{\bm{\weightAndBias}_{\iter - 1}; \mathcal{I}_{\iter}} + \regConst \bm{\weightAndBias}_{\iter-1}$
        \State $\bm{m}_{\iter} = \optimEmaConst_{1} \bm{m}_{\iter-1} + \lr{1 - \optimEmaConst_{1}} \bm{g}_{\iter}$
        \State $\bm{v}_{\iter} = \optimEmaConst_{2} \bm{v}_{\iter-1} + \lr{1 - \optimEmaConst_{2}} \bm{g}_{\iter} \elemMul \bm{g}_{\iter}$
        \State $\tilde{\bm{m}}_{\iter} = \bm{m}_{\iter} / \lr{1 - \optimEmaConst_{1}^{\iter}}$
        \State $\tilde{\bm{v}}_{\iter} = \bm{v}_{\iter} / \lr{1 - \optimEmaConst_{2}^{\iter}}$
        \State $\bm{\weightAndBias}_{\iter} = \bm{\weightAndBias}_{\iter-1} - \learningRate \tilde{\bm{m}}_{\iter} / \lr{\sqrt{\tilde{\bm{v}}_{\iter}} + \epsilon}$
        \EndFor
        \State \textbf{Return} $\bm{\weightAndBias}_{\iter}$
    \end{algorithmic}
\end{algorithm}
ここで,Adamでは正則化としてL2正則化を採用したが,Weight Decayを採用した最適化手法としてAdamW\cite{loshchilov2017decoupled}がある.AdamWの計算過程をアルゴリズム\ref{sec3:algo:adamw}に示す.正則化がL2正則化からWeight Decayに変わった点以外は同じである.
\begin{algorithm}
    \caption{AdamW}
    \label{sec3:algo:adamw}
    \begin{algorithmic}[1]
        \State \textbf{Input:} $\learningRate$, $\optimEmaConst_{1}$, $\optimEmaConst_{2}$, $\regConst$, $\bm{\weightAndBias}_{0}$, $L\lr{\bm{\weightAndBias}}$
        \State $\bm{m}_{0} = 0$, $\bm{v}_{0} = 0$
        \For{$\iter = 1$ to \texttt{...}}
        \State $\bm{g}_{\iter} = \nabla_{\bm{\weightAndBias}} \mathcal{\lossFuncUpper}\lr{\bm{\weightAndBias}_{\iter - 1}; \mathcal{I}_{\iter}}$
        \State $\bm{m}_{\iter} = \optimEmaConst_{1} \bm{m}_{\iter - 1} + \lr{1 - \optimEmaConst_{1}} \bm{g}_{\iter}$
        \State $\bm{v}_{\iter} = \optimEmaConst_{2} \bm{v}_{\iter - 1} + \lr{1 - \optimEmaConst_{2}} \bm{g}_{\iter} \elemMul \bm{g}_{\iter}$
        \State $\tilde{\bm{m}}_{\iter} = \bm{m}_{\iter} / \lr{1 - \optimEmaConst_{1}^{\iter}}$
        \State $\tilde{\bm{v}}_{\iter} = \bm{v}_{\iter} / \lr{1 - \optimEmaConst_{2}^{\iter}}$
        \State $\bm{\weightAndBias}_{\iter} = \bm{\weightAndBias}_{\iter - 1} - \learningRate \tilde{\bm{m}}_{\iter} / \lr{\sqrt{\tilde{\bm{v}}_{\iter}} + \epsilon} - \learningRate \regConst \bm{\weightAndBias}$
        \EndFor
        \State \textbf{Return} $\bm{\weightAndBias}_{\iter}$
    \end{algorithmic}
\end{algorithm}

\subsubsection{学習率のスケジューリング}
学習率のスケジューリングは,学習率$\learningRate$の値自体を学習の進行に伴って変更するものである.これは,より安定した学習を促したり,より早く学習を収束させたりするのに役立つ手段である.以下,3つのスケジューラを例として述べる.また,各スケジューラを用いた場合における学習率の遷移を図~\ref{sec3:fig:lr_scheduler}に示す.

1つ目は,StepLRSchedulerである.これは,初期学習率を$\learningRate_{0}$として,エポック$\epoch$における学習率$\learningRate_{\epoch}$を
\begin{equation}
    \learningRate_{\epoch} = \learningRate_{0} \gamma^{\left\lfloor \epoch / \epoch_{\text{step}} \right\rfloor}
\end{equation}
で与えるスケジューラである.これは,学習が$\epoch_{\text{step}}$エポック進むごとに学習率を$\gamma$倍することで,学習率を段階的に変化させる.シンプルで分かりやすいが,学習率の変化が不連続的になる.

2つ目は,ExponentialLRSchedulerである.これは,$\learningRate_{\epoch}$を
\begin{equation}
    \learningRate_{\epoch} = \learningRate_{0} \exp \lr{ -\gamma \epoch }
\end{equation}
で与えるスケジューラである.学習が1エポック進むごとに学習率を指数関数的に変化させるため,StepLRSchedulerと比較して変化が連続的である.

3つ目は,Cosine Annealing with Warmupである.これは,$\learningRate_{\epoch}$を
\begin{equation}
    \learningRate_{\epoch} =
    \begin{cases}
        \learningRate_{\text{min}} + \lr{ \frac{\epoch}{\epoch_{\text{warmup}}} } \lr{\learningRate_{\text{max}} - \learningRate_{\text{min}}}                                                                                    & \text{if $\epoch < \epoch_{\text{warmup}}$}   \\
        \learningRate_{\text{min}} + \frac{1}{2} \lr{\learningRate_{\text{max}} - \learningRate_{\text{min}}} \lr{ 1 + \cos \lr{ \frac{\lr{\epoch - \epoch_{\text{warmup}}}\pi}{\epoch_{\text{max}} - \epoch_{\text{warmup}}} } } & \text{if $\epoch \ge \epoch_{\text{warmup}}$}
    \end{cases}
\end{equation}
で与えるスケジューラである.ここで,$\learningRate_{\text{min}}$は最小学習率,$\learningRate_{\text{max}}$は最大学習率,$\epoch_{\text{max}}$は最大エポックである.$\epoch_{\text{warmup}}$は学習率を$\learningRate_{\text{min}}$から$\learningRate_{\text{max}}$まで線形に増加させるのにかけるエポック数を指定するパラメータである.エポック数が$\epoch_{\text{warmup}}$以上となれば,$\cos$関数に従って学習率を減衰させる.Cosine Annealing with Warmupは不安定になりがちな学習初期に学習率が低い状態から開始して,徐々に学習率を大きくすることで解の十分な探索を可能にし,その後再び学習率を小さくすることで学習の収束を促すスケジューラである.

\begin{figure}[bt]
    \centering
    \includegraphics[height=70mm]{./figure/sec3/lr_scheduler.png}
    \caption{スケジューラによる学習率の変化}
    \label{sec3:fig:lr_scheduler}
\end{figure}

\subsubsection{誤差逆伝播法}
\label{sec3:sec:backpropagation}
誤差逆伝播法は,DNNの各重みについての損失関数の勾配を,出力から入力へと遡る方向に計算するアルゴリズムである.ここでは例として,全結合層と活性化関数のみからなる$\numUpper_{\text{layer}}$層のDNNを構築し,ミニバッチ勾配降下法によって最適化する場面を考える\cite{higham2019deep}.また,\ref{sec3:sec:gradient_descent}節で定義した表記を再度用いる.

まず,$\weightLower_{p, q}^{n}$を$n - 1$層目の$p$番目のニューロンから$n$層目の$q$番目のニューロンに割り当てられた重み,$b_{p}^{n}$を$n$層目の$p$番目のニューロンに割り当てられたバイアスとすると,$n$層目の$p$番目のニューロンにおける出力$a_{p}^{n}$は,
\begin{equation}
    \label{sec3:eq:output_before_act}
    a_{p}^{n} = b_{p}^{n} + \sum_{q = 1}^{\dimUpper_{n - 1}} \weightLower_{q, p}^{n} o_{q}^{n - 1}
\end{equation}
で与えられる.ここで,$\dimUpper_{n - 1}$は$n - 1$層目の全結合層の次元(総ニューロン数),$o_{q}^{n - 1}$は$n - 1$層目の$q$番目のニューロンにおける出力$a_{q}^{n - 1}$に活性化関数$\phi$を適用した結果を表す.すなわち,
\begin{equation}
    o_{q}^{n - 1} = \phi\lr{a_{q}^{n - 1}}
\end{equation}
である.例外として,各$i \in \mathcal{I}$に対し,$\bm{o}^{0} = \bm{x}_{i}$とする.また,$\dimUpper_{0} = \dimUpper_{\text{in}}, \dimUpper_{N_{\text{layer}}} = \dimUpper_{\text{out}}$とする.ここで,式~\eqref{sec3:eq:output_before_act}に対し,$w_{0, p}^{n} = b_{p}^{n}$,$o_{0}^{n - 1} = 1$とおけば,
\begin{equation}
    \label{sec3:eq:output_before_act_2}
    a_{p}^{n} = b_{p}^{n} + \sum_{q = 1}^{\dimUpper_{n - 1}} w_{q, p}^{n} o_{q}^{n - 1}
    = \sum_{q = 0}^{\dimUpper_{n - 1}} w_{q, p}^{n} o_{q}^{n - 1}
\end{equation}
と整理できる.この時,
\begin{align}
    \pdv{\mathcal{\lossFuncUpper}\lr{\bm{\weightAndBias}; \mathcal{\indexUpper}}}{w_{p, q}^{n}} & = \pdv{w_{p, q}^{n}} \lr{\frac{1}{\card{\mathcal{\indexUpper}}} \sum_{\indexLower \in \mathcal{\indexUpper}} \lossFuncUpper\lr{\bm{\outputLower}_{\indexLower}, f\lr{\bm{\inputLower}_{\indexLower}; \bm{\weightAndBias}}}} \\
                                                                                                & = \frac{1}{\card{\mathcal{\indexUpper}}} \sum_{\indexLower \in \mathcal{\indexUpper}} \pdv{\lossFuncUpper\lr{\bm{\outputLower}_{\indexLower}, f\lr{\bm{\inputLower}_{\indexLower}; \bm{\weightAndBias}}}}{w_{p, q}^{n}}     \\
                                                                                                & = \frac{1}{\card{\mathcal{\indexUpper}}} \sum_{\indexLower \in \mathcal{\indexUpper}} \pdv{\lossFuncUpper_{\indexLower}}{w_{p, q}^{n}}
\end{align}
となる.ここで,$\lossFuncUpper\lr{\bm{\outputLower}_{\indexLower}, f\lr{\bm{\inputLower}_{\indexLower}; \bm{\weightAndBias}}} = \lossFuncUpper_{\indexLower}$とおいた.この時,各$n \in \lrc{1, \ldots, N_{\text{layer}}}$に対し,
$\pdv*{\lossFuncUpper_{\indexLower}}{w_{p, q}^{n}}$は式\eqref{sec3:eq:output_before_act_2}を用いて,
\begin{align}
    \pdv{\lossFuncUpper_{\indexLower}}{w_{p, q}^{n}} & = \pdv{\lossFuncUpper_{\indexLower}}{a_{q}^{n}} \pdv{a_{q}^{n}}{w_{p, q}^{n}}                \\
                                                     & = \delta_{q}^{n} \pdv{w_{p, q}^{n}} \lr{\sum_{r = 0}^{D_{n - 1}} w_{r, q}^{n} o_{r}^{n - 1}} \\
                                                     & = \delta_{q}^{n} o_{p}^{n - 1}
\end{align}
となる.ここで,$\pdv*{\lossFuncUpper_{\indexLower}}{a_{q}^{n}} = \delta_{q}^{n}$とおいた.このとき,最終層($n = \numUpper_{\text{layer}}$)の重みの場合,
\begin{align}
    \pdv{L_{\indexLower}}{w_{p, q}^{\numUpper_{\text{layer}}}} & = \delta_{q}^{\numUpper_{\text{layer}}} o_{p}^{\numUpper_{\text{layer}} - 1}                                                                                                                                                                                                           \\
                                                               & = o_{p}^{\numUpper_{\text{layer}} - 1} \pdv{L_{i}}{a_{q}^{\numUpper_{\text{layer}}}}                                                                                                                                                                                                   \\
                                                               & = o_{p}^{\numUpper_{\text{layer}} - 1} \pdv{\lossFuncUpper\lr{\bm{\outputLower}_{\indexLower}, f\lr{\bm{\inputLower}_{\indexLower}; \bm{\weightAndBias}}}}{a_{q}^{\numUpper_{\text{layer}}}}                                                                                           \\
                                                               & = o_{p}^{\numUpper_{\text{layer}} - 1} \pdv{\lossFuncUpper\lr{\bm{\outputLower}_{\indexLower}, \phi\lr{\bm{a}^{\numUpper_{\text{layer}}}}}}{a_{q}^{\numUpper_{\text{layer}}}}                                                                                                          \\
                                                               & = o_{p}^{\numUpper_{\text{layer}} - 1} \lr{\nabla_{\phi\lr{\bm{a}^{\numUpper_{\text{layer}}}}} \lossFuncUpper\lr{\bm{\outputLower}_{\indexLower}, \phi\lr{\bm{a}^{\numUpper_{\text{layer}}}}}}^\top \pdv{\phi\lr{\bm{a}^{\numUpper_{\text{layer}}}}}{a_{q}^{\numUpper_{\text{layer}}}} \\
                                                               & = o_{p}^{\numUpper_{\text{layer}} - 1} \pdv{\lossFuncUpper\lr{\bm{\outputLower}_{\indexLower}, \phi\lr{\bm{a}^{\numUpper_{\text{layer}}}}}}{\phi\lr{a_{q}^{\numUpper_{\text{layer}}}}} \phi'\lr{a_{q}^{\numUpper_{\text{layer}}}}
\end{align}
となる.これは,入力から出力を計算する順伝搬で得られた値のみに依存するから,直ちに計算可能であることがわかる.一方,最終層以外($1 \le n < \numUpper_{\text{layer}}$)の重みの場合,
\begin{align}
    \pdv{L_{\indexLower}}{w_{p, q}^{n}} & = \delta_{q}^{n} o_{p}^{n - 1}                                                                                                            \\
                                        & = o_{p}^{n - 1} \pdv{L_{i}}{a_{q}^{n}}                                                                                                    \\
                                        & = o_{p}^{n - 1} \sum_{r = 0}^{D_{n + 1}} \pdv{L_{\indexLower}}{a_{r}^{n + 1}} \pdv{a_{r}^{n + 1}}{a_{q}^{n}}                              \\
                                        & = o_{p}^{n - 1} \sum_{r = 0}^{D_{n + 1}} \delta_{r}^{n + 1} \pdv{a_{q}^{n}} \lr{\sum_{s = 0}^{D_{n}} w_{s, r}^{n + 1} o_{s}^{n}}          \\
                                        & = o_{p}^{n - 1} \sum_{r = 0}^{D_{n + 1}} \delta_{r}^{n + 1} \pdv{a_{q}^{n}} \lr{\sum_{s = 0}^{D_{n}} w_{s, r}^{n + 1} \phi\lr{a_{s}^{n}}} \\
                                        & = o_{p}^{n - 1} \sum_{r = 0}^{D_{n + 1}} \delta_{r}^{n + 1} w_{q, r}^{n + 1} \phi'\lr{a_{q}^{n}}                                          \\
                                        & = o_{p}^{n - 1} \phi'\lr{a_{q}^{n}} \sum_{r = 0}^{D_{n + 1}} \delta_{r}^{n + 1} w_{q, r}^{n + 1}
\end{align}
となる.これは,順伝搬時には計算されない$\delta_{r}^{n + 1}$に依存しているから,$n + 1$層目についての勾配計算を先に行う必要があることがわかる.従って,最終層のみ直ちに勾配を計算可能であり,それ以外の層は自身の次の層に依存しているから,出力から入力へとDNNを遡る方向に計算する,誤差逆伝播法が効率の良いアルゴリズムだと言える.

\subsubsection{学習の安定化}
DNNの学習は勾配降下法によって行われるが,ここで勾配が大きくなりすぎると重みの更新幅が過剰に大きくなり,学習が不安定になる可能性がある.これに対して,Gradient Clippingが有効である.これは,
\begin{equation}
    \nabla_{\bm{\weightAndBias}} \mathcal{\lossFuncUpper}\lr{\bm{\weightAndBias}; \mathcal{I}} \gets \frac{c}{\max \lrc{\lrTwoNorm{\nabla_{\bm{\weightAndBias}} \mathcal{\lossFuncUpper}\lr{\bm{\weightAndBias}; \mathcal{I}}}, c}} \nabla_{\bm{\weightAndBias}} \mathcal{\lossFuncUpper}\lr{\bm{\weightAndBias}; \mathcal{I}}
\end{equation}
で与えられる.ここで,$c$は勾配のL2ノルムに対する閾値である.$\mathcal{\lossFuncUpper}\lr{\bm{\weightAndBias}; \mathcal{I}}$は\ref{sec3:sec:gradient_descent}節で定義した関数を再度用いた.

また,近年は数億単位のパラメータを持つ大規模なDNNも提案されており,こういった規模間のDNNを構築して学習する場合,それ相応のメモリが必要になる.マシンのスペックに対し,バッチサイズを十分小さくすれば基本的に学習は可能であるが,これは各データのノイズの影響が強くなるため,学習を不安定にする要因となる.これに対し,Gradient Accumulationが有効である.Gradient Accumulationは,小さなバッチサイズで計算した勾配を複数イテレーションに渡って累積し,設定したイテレーション数ごとに重みの更新を行う手法である.累積される勾配を$\bm{g}_{\text{accum}}$とすると,この更新は
\begin{equation}
    \bm{g}_{\text{accum}} \gets \bm{g}_{\text{accum}} + \nabla_{\bm{\weightAndBias}} \mathcal{\lossFuncUpper}\lr{\bm{\weightAndBias}_{\iter - 1}; \mathcal{I}_{\iter}}
\end{equation}
で与えられる.ここで,設定した累積回数を$\numUpper_{\text{accum}}$とすると,重み$\bm{\weightAndBias}$の更新は
\begin{equation}
    \bm{\weightAndBias}_{\iter} = \bm{\weightAndBias}_{\iter - 1} - \frac{\learningRate}{\numUpper_{\text{accum}}} \bm{g}_{\text{accum}}
\end{equation}
で与えられる.$\numUpper_{\text{accum}}$回分の勾配を累積した分,重みを更新する際には$1 / \numUpper_{\text{accum}}$倍して平均をとることで,実質的に$\numUpper_{\text{accum}}$倍のバッチサイズにおける学習が可能になる.また,重み更新後は累積した勾配を0にリセットして,次の$\numUpper_{\text{accum}}$回の累積に備える.

% \subsubsection{自己教師あり学習}
% \label{sec3:sec:ssl}
% 近年,音声や動画を用いる分野では,自己教師あり学習を事前に行ったモデルを特定の問題にFineTuningする転移学習の有効性が確認されている.自己教師あり学習とは、教師ラベルのないデータから特徴を学習する手法であり、データ自体を利用して擬似的な教師ラベルを生成し、教師あり学習を行う点が特徴である。このアプローチは教師なし学習と類似しているが、教師ラベルを生成して利用する点で教師あり学習に近いといえる。ここでは特に,本研究で用いる自己教師あり学習モデルであるHuBERT\cite{hsu2021hubert},AVHuBERT\cite{shi2022learning}で行われている,Masked Predictionという自己教師あり学習方法について述べる.

% Masked Predictionでは,マスクによって一部欠損された入力から,マスクされたフレームにおける教師ラベルを予測することで学習を行う.モデルへの入力を$\bm{\inputUpper} \in \realSet^{\timeUpper \times \dimUpper}$とし,このうちマスクされるインデックスの部分集合を$\mathcal{M} \subset \lrc{1, \ldots \timeUpper}$とする.$T$は系列長,$D$は特徴量の次元である.この時,マスクされた入力$\bm{\inputUpper}^{\text{masked}}$の各時刻$\timeLower$におけるベクトル$\bm{\inputLower}^{\text{masked}}_{\timeLower}$は,
% \begin{equation}
%     \bm{\inputLower}^{\text{masked}}_{\timeLower} =
%     \begin{cases}
%         \bm{\inputLower}_{\timeLower} & \text{if $\timeLower \notin \mathcal{M}$} \\
%         \bm{m}                        & \text{if $\timeLower \in \mathcal{M}$}
%     \end{cases}
% \end{equation}
% となる.$\bm{m} \in \realSet^{\dimUpper}$はマスク専用のベクトルである.ここで,HuBERTでは音声データ,AVHuBERTでは動画データと音声データを扱うが,いずれもクラスタリングによって連続特徴量を離散化し,教師ラベルを作成する.クラスタ数$\classUpper$のクラスタリングによって得られた教師ラベルを$\bm{\outputUpper} \in \lrc{0, 1}^{\timeUpper \times \classUpper}$とする.各時刻$\timeLower$における教師ラベル$\bm{y}_{\timeLower}$は,正解クラスが1,それ以外が0となったOne-hotベクトルである.この時,モデルからのロジット$\hat{\bm{\outputUpper}} \in \realSet^{\timeUpper \times \classUpper}$に対する損失は,
% \begin{equation}
%     \lossFuncUpper_{\text{CE-masked}}\lr{\bm{\outputUpper}, \hat{\bm{\outputUpper}}} =
%     - \frac{1}{\card{\mathcal{M}}} \sum_{\timeLower \in \mathcal{M}} \sum_{c = 1}^{\classUpper} y_{t, c} \log \lr{\frac{\exp\lr{\hat{\outputLower}_{\timeLower, \classLower}}}{\sum_{\indexLower = 1}^{C} \exp\lr{\hat{\outputLower}_{\timeLower, \indexLower}}}}
% \end{equation}
% で与えられる.すなわち,マスクされた位置に限定したCross Entropy Lossである.この損失関数の値を最小化するためには,欠損された入力から特徴抽出を行う必要があるから,最適化されたDNNは音声や動画における文脈的な構造を学習していると考えられる.また,Masked Predictionはテキストアノテーションを必要としない学習方法であるから,教師あり学習と比較してより多くのリソースを用いた学習が可能になるという利点がある.


\clearpage

\section{動画音声合成モデルの検討}
\subsection{音声合成法}
\subsubsection{全体像}
本実験で用いる学習データセット$\datasetTrain$を
\begin{equation}
    \datasetTrain = \lrc{\lr{\video, \spWaveformGt, \spkEmb, \melGt, \hubertIntGt, \hubertDiscGt}}_{\numLower = 1}^{\numUpper}
\end{equation}
とする.各$\numLower$に対し,$\video \in \videoSet$は口唇動画, $\spWaveformGt \in \spWaveformSet$は原音声の音声波形,$\spkEmb \in \spkEmbSet$は話者ベクトル,$\melGt \in \melSet$はメルスペクトログラム,$\hubertIntGt \in \hubertIntSet$はHuBERT中間特徴量,$\hubertDiscGt \in \hubertDiscGtSet$はHuBERT離散特徴量とする.ここで,話者ベクトル$\spkEmb$は,話者識別モデル\cite{wan2018generalized}を利用して得られた音声の話者性を反映するベクトル(d-vector)であり,
\begin{equation}
    \spkEmb = \frac{1}{|\mathcal{\indexUpper}|} \sum_{\indexLower \in \mathcal{\indexUpper}} \spkEmbExtractor\lr{\bm{\outputLower}^{\text{sp-wf}}_{\indexLower}; \weightSpk}
\end{equation}
で与えられる.ここで,$\weightSpk$は事前学習済み重みを表す.$\mathcal{\indexUpper}$は学習データセット$\datasetTrain$において、話者$s_{n}$と話者が同一であるデータのインデックス集合から、ランダムに$\numUpper_{\text{spk-emb}}$個のインデックスを抽出した部分集合とする。すなわち、
\begin{equation}
    \mathcal{\indexUpper} \subset \{\indexLower \mid \indexLower \in \{1, \ldots, N\}, ~ s_{\indexLower} = \spkId\}, \quad |\mathcal{\indexUpper}| = \numUpper_{\text{spk-emb}}
\end{equation}
である。次に,HuBERT中間特徴量とHuBERT離散特徴量は,HuBERTにおける畳み込みエンコーダまで(Transformer層以前)を$\hubertConv$,HuBERTにおけるTransformer層($\hubertConv$以後)を$\hubertTransformer$,k-means法を$\kmeans$,インデックス系列をOne-hotベクトルに変換する関数を$\onehot$とすると,
\begin{gather}
    \hubertIntGt = \hubertConv\lr{\spWaveformGt; \weightHuBERTConv} \\
    \hubertDiscGt = \onehot\lr{\kmeans\lr{\hubertTransformer\lr{\hubertIntGt; \weightHuBERTTrans}}}
\end{gather}
で与えられる.ここで,$\weightHuBERTConv, \weightHuBERTTrans$はHuBERTの事前学習済み重みを表す.HuBERTを利用した特徴量計算の流れを図\ref{sec4:fig:hubert}に示す。

\begin{figure}[b]
    \centering
    \includegraphics[height=12mm]{./figure/sec4/model_2/hubert.drawio.png}
    \caption{HuBERTを利用した特徴量計算の流れ}
    \label{sec4:fig:hubert}
\end{figure}

提案手法を図\ref{sec4:fig:overview}に示す.提案手法は,ネットワークA,ネットワークB,ボコーダの三つからなる.まず,ネットワークAを$\myNetworkA$とすると,$\myNetworkA$の行う処理は,
\begin{equation}
    \label{sec4:eq:networkA_overview}
    \hubertIntPred, \melPredA, \hubertDiscPredA = \myNetworkA\lr{\video, \spkEmb; \weightA}
\end{equation}
と表される.ここで,$\hubertIntPred \in \hubertIntSet$は予測HuBERT中間特徴量,$\melPredA \in \melSet$はネットワークAの予測メルスペクトログラム,$\hubertDiscPredA \in \hubertDiscPredSet$はネットワークAのHuBERT離散特徴量に対するロジットを表す.次に,ネットワークBを$\myNetworkB$とすると,$\myNetworkB$の行う処理は,
\begin{equation}
    \melPredB, \hubertDiscPredB = \myNetworkB\lr{\hubertIntPred, \spkEmb; \weightB}
\end{equation}
と表される.ここで,$\melPredB \in \melSet$はネットワークBの予測メルスペクトログラム,$\hubertDiscPredB \in \hubertDiscPredSet$はネットワークBのHuBERT離散特徴量に対するロジットを表す.最後に,音声波形を生成するボコーダを$\vocoder$とすると,$\vocoder$の行う処理は,
\begin{equation}
    \spWaveformPred = \vocoder\lr{\melPredB, \hubertDiscPredB; \weightVoc}
\end{equation}
と表される.まとめると,提案手法は口唇動画と話者ベクトルを入力とし,ネットワークAとネットワークBによって中間表現を獲得後,中間表現をボコーダに入力することで音声波形を生成するものである.

\begin{figure}[bt]
    \centering
    \includegraphics[height=100mm]{./figure/sec4/model_2/overview.drawio.png}
    \caption{提案手法の全体像}
    \label{sec4:fig:overview}
\end{figure}

\subsubsection{ネットワークA}
ネットワークAを図\ref{sec4:fig:networkA}に示す。ネットワークAでは,まず,口唇動画$\video$をAVHuBERTに通すことで,特徴量$\featureA \in \featureASet$に変換する.これは,
\begin{equation}
    \featureA = \AVHuBERT\lr{\video; \weightAAVHuBERT}
\end{equation}
と表される.$\weightAAVHuBERT$は,AVHuBERTの事前学習済み重みで初期化した.次に,$\featureA$の各時刻$t$におけるベクトルに対し,話者ベクトル$\spkEmb$を次元方向に結合してから,全結合層によって次元を再度圧縮し,元の次元に戻す.これは,
\begin{equation}
    \featureA = \myNetworkSpkMerge\lr{\concat{\featureA, \lrRepeat{\spkEmb}{T}}; \weightAFcSpk}
\end{equation}
と表される.次に,話者ベクトルが統合された特徴量に対する後処理として,$\myNetworkPost$を適用する.これは,
\begin{equation}
    \featureA = \myNetworkPost\lr{\featureA; \weightAPost}
\end{equation}
と表される.ここで、後処理層$\myNetworkPost$の構造を図\ref{sec4:fig:post_three_step}に示す。$\myNetworkPost$は,一次元畳み込み層を主としたConvBlockおよび,これに残差結合を組み合わせたResBlockから構成される.$\myNetworkPost$は,話者ベクトル$\spkEmb$が次元方向に結合された特徴量$\featureA$に対する話者性を考慮した特徴量の変換を行うために導入した。$\AVHuBERT$はMasked Predictionによって事前学習されたモデルだから、動画の文脈的な構造を考慮するのに適していると考えられる一方で、音声合成に必要となる話者性は発話に依存しない、すなわち動画の文脈的な構造とは別の性質を持った情報だと考えたからである.最後に,$\featureA$を全結合層を通して変換することで,予測対象であるHuBERT中間特徴量,メルスペクトログラム,HuBERT離散特徴量に対するロジットを得る.これは,
\begin{gather}
    \hubertIntPred = \myNetworkFcHubInt\lr{\featureA; \weightAFcHubInt} \\
    \melPredA = \myNetworkFcMel\lr{\featureA; \weightAFcMel} \\
    \hubertDiscPredA = \myNetworkFcHubDisc\lr{\featureA; \weightAFcHuBDisc}
\end{gather}
と表される.$\weightAFcSpk, \weightAPost, \weightAFcHubInt, \weightAFcMel, \weightAFcHuBDisc$は,すべてランダムに初期化した.

ネットワークAの役割は,続くネットワークBの入力であるHuBERT中間特徴量を予測することである.これに対し,メルスペクトログラムとHuBERT離散特徴量の推定を同時に行った理由は,先行研究\cite{kim2023lip_multitask,choi2023intelligible}においてマルチタスク学習の有効性が確認されており、HuBERT中間特徴量の推定においても有効ではないかと考えたからである。

\begin{figure}[tb]
    \centering
    \begin{subfigure}[b]{0.48\textwidth}
        \centering
        \includegraphics[height=120mm]{./figure/sec4/model_2/networkA.drawio.png}
        \caption{ネットワークA}
        \label{sec4:fig:networkA}
    \end{subfigure}
    \hfill
    \begin{subfigure}[b]{0.48\textwidth}
        \centering
        \includegraphics[height=120mm]{./figure/sec4/model_2/networkB.drawio.png}
        \caption{ネットワークB}
        \label{sec4:fig:networkB}
    \end{subfigure}
    \hfill
    \caption{ネットワークAとネットワークBの構造}
    \label{sec4:fig:networkAB}
\end{figure}

\begin{figure}[tb]
    \centering
    \begin{subfigure}[b]{0.32\textwidth}
        \centering
        \includegraphics[height=70mm]{./figure/sec4/model_2/post.drawio.png}
        \caption{全体}
        \label{sec4:fig:post}
    \end{subfigure}
    \hfill
    \begin{subfigure}[b]{0.32\textwidth}
        \centering
        \includegraphics[height=70mm]{./figure/sec4/model_2/post_resblock.drawio.png}
        \caption{$\text{ResBlock}$}
        \label{sec4:fig:post_resblock}
    \end{subfigure}
    \hfill
    \begin{subfigure}[b]{0.32\textwidth}
        \centering
        \includegraphics[height=70mm]{./figure/sec4/model_2/post_convblock.drawio.png}
        \caption{$\text{ConvBlock}$}
        \label{sec4:fig:post_convblock}
    \end{subfigure}
    \caption{後処理層$\myNetworkPost$の構造}
    \label{sec4:fig:post_three_step}
\end{figure}

\subsubsection{ネットワークB}
ネットワークBを図\ref{sec4:fig:networkB}に示す。ネットワークBでは,まず,ネットワークAで得られた予測HuBERT中間特徴量$\hubertIntPred$を$\hubertTransformer$に通すことで,特徴量$\featureB \in \featureBSet$に変換する.これは,
\begin{equation}
    \featureB = \hubertTransformer\lr{\hubertIntPred; \weightBHuBERTTrans}
\end{equation}
と表される.$\weightBHuBERTTrans$は,HuBERTの事前学習済み重みで初期化する場合と,ランダム初期化する場合の二つを検討した.以下,ネットワークAと処理は同様であるため,数式のみ記載する.
\begin{gather}
    \featureB = \myNetworkSpkMerge\lr{\concat{\featureB, \lrRepeat{\spkEmb}{T}}; \weightBFcSpk} \\
    \featureB = \myNetworkPost\lr{\featureB; \weightBPost} \\
    \melPredB = \myNetworkFcMel\lr{\featureB; \weightBFcMel} \\
    \hubertDiscPredB = \myNetworkFcHubDisc\lr{\featureB; \weightBFcHuBDisc}
\end{gather}
$\weightBFcSpk, \weightBPost, \weightBFcMel, \weightBFcHuBDisc$は,すべてランダムに初期化した.

ネットワークBの役割は,メルスペクトログラムとHuBERT離散特徴量の予測である.HuBERT Transformer層の転移学習を検討した狙いは、HuBERTの自己教師あり学習の方法に基づく。HuBERTはMasked Predictionという自己教師あり学習を行っており、これは、畳み込みエンコーダ$\hubertConv$からの出力にマスクを適用し,Transformer層$\hubertTransformer$を通すことによって、マスクされたフレームにおける教師ラベルを推定する学習方法である.\ref{sec3:sec:ssl}節より、Masked Predictionでは欠損された入力から特徴抽出を行う必要があるから,最適化されたHuBERTは音声の文脈的な構造を学習していると考えられる.また、Masked Predictionは音声データのみによる学習が可能であり、近年では大規模なデータによって学習されたモデルも一般に公開されている。よって、本研究ではHuBERT Transformer層を動画音声合成にFine Tuningすることにより,動画を入力とするネットワークAの推定残差を,大量の音声データから学習された音声自体の文脈を考慮する力によって軽減することで、最終予測値であるメルスペクトログラムとHuBERT離散特徴量に対する推定精度改善を狙った.

\subsubsection{ボコーダ}
ボコーダを図\ref{sec4:fig:vocoder}に示す。本実験で用いるボコーダは、今回ベースラインとする先行研究\cite{choi2023intelligible}で提案されたMulti-input Vocoderを参考にしたものである。まず,ネットワークBで得られた予測メルスペクトログラム$\melPredB$とHuBERT離散特徴量に対するロジット$\hubertDiscPredB$を前処理層に通し,扱いやすい形状に変換する.これは,
\begin{gather}
    \featureVocMel = \vocoderPreMel\lr{\melPredB; \weightVocPreMel} \\
    \featureVocHuBERT = \vocoderPreHub\lr{\hubertDiscPredB; \weightVocPreHuB}
\end{gather}
と表される.$\featureVocMel \in \featureVocMelSet$は,$\melPredB$の時間方向に隣接したベクトルを次元方向に結合することで$\hubertDiscPredB$と系列長を揃え、その後全結合層を適用した特徴量である.一方,$\featureVocHuBERT \in \featureVocHuBERTSet$は,$\hubertDiscPredB$に対して$\argmax$関数を適用した後,各時刻$t$において選択されたインデックスをベクトルに変換した特徴量である.次に,$\featureVocMel, \featureVocHuBERT$を入力として,音声波形を生成する.これは,
\begin{equation}
    \spWaveformPred = \vocoderMain\lr{\concat{\featureVocMel, \featureVocHuBERT}; \weightVocMain}
\end{equation}
と表される.$\vocoderMain$は,図\ref{sec4:fig:vocoder_main}に示すように,一次元畳み込み層とUpsamplingBlockから構成され,これはHiFi-GANのGeneratorと同様である.各UpsamplingBlockでは,初めに一次元転置畳み込み層を通すことで時間方向のアップサンプリングを行い,その後複数の一次元畳み込み層から特徴抽出を行った結果を平均して出力する.

\begin{figure}[tb]
    \centering
    \begin{subfigure}[b]{0.32\textwidth}
        \centering
        \includegraphics[width=45mm]{./figure/sec4/model_2/vocoder.drawio.png}
        \caption{全体}
        \label{sec4:fig:vocoder_overview}
    \end{subfigure}
    \hfill
    \begin{subfigure}[b]{0.32\textwidth}
        \centering
        \includegraphics[width=45mm]{./figure/sec4/model_2/vocoder_main.drawio.png}
        \caption{$\vocoderMain$}
        \label{sec4:fig:vocoder_main}
    \end{subfigure}
    \hfill
    \begin{subfigure}[b]{0.32\textwidth}
        \centering
        \includegraphics[width=45mm]{./figure/sec4/model_2/vocoder_main_block.drawio.png}
        \caption{$\text{UpsamplingBlock}$}
        \label{sec4:fig:vocoder_main_block}
    \end{subfigure}
    \caption{ボコーダの構造}
    \label{sec4:fig:vocoder}
\end{figure}

\subsubsection{損失関数}
まず,ネットワークAの学習に用いる損失関数$\lossA$は,
\begin{equation}
    \begin{aligned}
        \lossA & \lr{\hubertIntGt, \hubertIntPred, \melGt, \melPredA, \hubertDiscGt, \hubertDiscPredA}                 \\
        =      & \lossWeightHubInt \lossMAE{\hubertIntGt}{\hubertIntPred} + \lossWeightMel \lossMAE{\melGt}{\melPredA} \\
               & + \lossWeightHubDisc \lossCE{\hubertDiscGt}{\hubertDiscPredA}
    \end{aligned}
\end{equation}
で与えられる.すなわち,HuBERT中間特徴量についてのMAE Loss,メルスペクトログラムについてのMAE Loss,HuBERT離散特徴量についてのCross Entropy Lossの重み付け和である.次に,ネットワークBの学習に用いる損失関数$\lossB$は,
\begin{equation}
    \begin{aligned}
        \lossB & \lr{\melGt, \melPredB, \hubertDiscGt, \hubertDiscPredB}                                                  \\
        =      & \lossWeightMel \lossMAE{\melGt}{\melPredB} + \lossWeightHubDisc \lossCE{\hubertDiscGt}{\hubertDiscPredB}
    \end{aligned}
\end{equation}
で与えられる.すなわち,メルスペクトログラムについてのMAE Loss,HuBERT離散特徴量についてのCross Entropy Lossの重み付け和である.最後に,ボコーダの学習に用いる損失関数について,これはHiFi-GANと同様であるから,ここでは割愛する.

\subsection{実験方法}
\subsubsection{利用したデータセット}
動画音声データセットには,先行研究\cite{taguchi,esaki}で収録されたもののうち,ATR音素バランス文\cite{atr}を読み上げたデータを利用した.このデータセットは、男女二人ずつから収録された合計4人分のデータから構成される。分割は,AからHセットを学習データ,Iセットを検証データ,Jセットをテストデータとした.各分割ごとの文章数を表~\ref{sec4:tab:dataset_info}に示す.

ボコーダの学習には,Hi-Fi-Captain\cite{okamoto2023hi}とJVS\cite{takamichi2019jvs}を利用した.Hi-Fi-Captainは,日本語話者2名と英語話者2名からなるデータセットであるが,本実験では日本語話者2名分のデータのみを利用した.分割は,train-parallelおよびtrain-non-parallelを学習データ,valを検証データ,evalをテストデータとした.各分割ごとの文章数を表~\ref{sec4:tab:dataset_info}に示す.JVSは100人の話者からなるデータであり,各話者に対して1から100まで番号が割り振られている.読み上げ音声のparallel100およびnonpara30と,裏声のfalset10,囁き声のwhisper10が含まれるが,本研究ではparallel100とnonpara30のみ利用した.分割は,1から80番の話者を学習データ,81番から90番の話者を検証データ,91番から100番までの話者をテストデータとした.各分割ごとの文章数を表~\ref{sec4:tab:dataset_info}に示す.

\begin{table*}[bt]
    \centering
    \caption{利用したデータセットの文章数}
    \label{sec4:tab:dataset_info}
    \begin{center}
        \renewcommand{\arraystretch}{0.9} % 行の高さ調整
        \setlength{\tabcolsep}{8pt}      % 列の幅調整
        \scalebox{1.0}{
            \begin{tabular}{|l|r|r|r|}
                \hline
                              & \multicolumn{1}{c|}{学習} & \multicolumn{1}{c|}{検証} & \multicolumn{1}{c|}{テスト} \\
                \hline
                動画音声データセット    & 1598                    & 200                     & 212                      \\
                Hi-Fi-Captain & 37714                   & 200                     & 200                      \\
                JVS           & 10398                   & 1299                    & 1300                     \\
                \hline
            \end{tabular}
        }
    \end{center}
\end{table*}

\subsubsection{データの前処理}
動画データは60 FPSで収録されたものをffmpegにより25 FPSに変換して用いた.その後,手法\cite{bulat2017far}により動画に対してランドマーク検出を適用した\footnote{\url{https://github.com/1adrianb/face-alignment}}.このランドマークを利用することで口元のみを切り取り,画像サイズを$\lr{96, 96}$にリサイズした.モデル入力時は動画をグレースケールに変換し,各フレームに対する正規化および標準化を適用した.全体として,今回はAVHuBERTの転移学習を行うため,そこでの前処理に合わせている.学習時のデータ拡張は,ランダムクロップ,左右反転,Time Maskingを適用した.ランダムクロップは,$\lr{96, 96}$で与えられる画像から$\lr{88, 88}$をランダムに切り取る処理である.検証およびテスト時は,必ず画像中央を切り取るよう実装した.左右反転は,50\%の確率で左右が反転されるよう実装した.Time Maskingは,動画1秒あたり0から0.5秒の間でランダムに停止区間を定め,その区間における動画の時間方向平均値を計算し,区間内のすべてのフレームをこの平均値で置換した.これにより,動画が一時停止されるような効果が得られる.

音声データは16 kHzにダウンサンプリングして用いた.窓長25 msのハニング窓を用いてシフト幅10 msでSTFTを適用することで,フレームレート100 Hzのスペクトログラムに変換し,パワースペクトログラムに対して80次のメルフィルタバンクを適用した後,対数スケールに変換することで対数メルスペクトログラムを得た.また,Hi-Fi-CaptainとJVSには,ボコーダの学習安定化のため,無音区間のトリミング(-40 dBFS未満かつ500 ms継続する区間を100 msまでカット)を適用した.

モデルへの入力とした話者ベクトルは,動画音声データセット,Hi-Fi-Captain,JVSともに,各話者に対し学習データの中から100文章をランダムサンプリングし,各発話に対して得られたベクトルの平均値を用いた.これを学習・検証・テストで一貫して用いるため,検証データやテストデータには非依存な値となっている.

HuBERT離散特徴量の計算に利用するHuBERT Transformer層出力は,8層目出力を利用した.HuBERTの層ごとの特徴量について,音素のOne-hotベクトルおよび単語のOne-hotベクトルとの相関をCanonical Correlation Analysis(CCA)によって調べた先行研究\cite{pasad2023comparative}より,8層目出力がそのどちらとも相関が高いことが示されている.本実験では,HuBERT離散特徴量は言語的な情報を持つものとして扱いたかったため,この層からの出力を利用した.k-means法のクラスタ数は100とし,動画音声データセットの学習用データを利用して学習した後,全データセットに対してクラスタリングを実施した.また,学習時はゼロパディングされる区間のためにクラスを一つ追加したため,合計101クラスとして扱った.

\subsubsection{本実験で利用した事前学習済みモデルについて}
話者ベクトルの計算に用いた話者識別モデルの事前学習済み重みには,VoxCeleb1\cite{nagrani2020voxceleb}とVoxCeleb2,LibriSpeech\cite{panayotov2015librispeech}のotherセットで学習されたものを用いた\footnote{\url{https://github.com/resemble-ai/Resemblyzer}}.学習データセットの記述は,このモデルを利用して複数話者TTSを検討した先行研究\cite{jia2018transfer}のGitHubリポジトリにある\footnote{\url{https://github.com/CorentinJ/Real-Time-Voice-Cloning/wiki/Training}}.話者識別モデルは,音声波形をメルスペクトログラムに変換した後,これを1.6秒ごとに0.77秒のオーバーラップを持つよう分割し,各区間ごとに一つのベクトル表現を獲得した後,区間ごとの出力ベクトルを平均して正規化することで最終出力を得る.モデル構造は3層のLSTMと全結合層からなり,得られる話者ベクトルの次元は256次元である.

AVHuBERTの事前学習済み重みには,LRS3とVoxCeleb2の英語データを利用し,データ拡張として音声データにノイズを付与した場合\cite{shi2022robust}の重みを利用した\footnote{\url{https://github.com/facebookresearch/av_hubert}}.脚注のリンク先では,「Model: Noise-Augmented AV-HuBERT Base」,「Pretraining Data: LRS3 + VoxCeleb2 (En)」,「Finetuning Data: No finetuning」と示されている.ノイズ付与の場合を利用した理由は,音声にノイズが含まれることで動画からの特徴抽出が促進され,より動画タスクに適した重みになっているのではないかと考えたからである.AVHuBERTは3次元畳み込み層と2次元畳み込み層を中心としたResNetと,12層のTransformer層から構成される.ResNetによって形状が$\lr{\dimUpper \times \timeUpper \times \widthUpper \times \heightUpper}$である動画からの特徴抽出および空間情報の圧縮が行われ,出力特徴量の形状は$\lr{\dimUpper \times \timeUpper}$となる.これに対してTransformer層を適用することで,時系列全体を考慮した特徴抽出を行う.最終的に得られる特徴量は768次元で,25 Hzである.

HuBERTの事前学習済み重みには,HuggingFaceに公開されているReazonSpeechによって学習されたもの\cite{rinna-japanese-hubert-base,sawada2024release}を利用した\footnote{\url{https://huggingface.co/rinna/japanese-hubert-base}}.ReazonSpeechは約19000時間の日本語音声からなるデータセットであり,今回用いる動画音声データセットが日本語であることから,このモデルが検討対象に適すると判断した.HuBERTは一次元畳み込み層を中心とした畳み込みエンコーダと,12層のTransformer層から構成される.畳み込みエンコーダによって音声波形の系列長を削減しつつ次元を上げた後,Transformer層を適用することで,時系列全体を考慮した特徴抽出を行う.最終的に得られる特徴量の次元は768次元で,50 Hzである.

\subsubsection{本実験で独自に構築したモデルについて}
本実験で独自に構築したモデルは,ネットワークAとBにおける$\myNetworkPost$と,ボコーダ$\vocoder$である.

$\myNetworkPost$について,ConvBlockにおける畳み込み層の次元は入力される特徴量に揃えて768次元とし,カーネルサイズは3とした.Dropoutは$p = 0.1$で用いた.ConvBlockから構成されるResBlockは3層積み重ねた.

ボコーダについて,メルスペクトログラムの前処理層$\vocoderPreMel$では,入力される80次元100 Hzのメルスペクトログラムに対し,時間方向に隣接した2フレームを次元方向に積むことで160次元50 Hzの特徴量に変換し,全結合層を適用して128次元まで次元を削減した.HuBERT離散特徴量の前処理層$\vocoderPreHub$では,初めに各時刻$t$におけるロジットに対して$\argmax$を適用することで,最も確率の高いクラスを選択する.ここで,学習時など原音声から計算されたインデックス系列が初めから入力される場合は,この処理をスキップする.その後,各時刻$t$におけるインデックスを128次元のベクトルに変換することで,128次元50 Hzの特徴量が得られる.前処理後の二つの特徴量を次元方向に結合することで,256次元50 Hzの特徴量が得られる.次に,メインの処理層$\vocoderMain$では,初めにカーネルサイズ7の畳み込み層により,次元を1024次元まで拡大する.これに対し,転置畳み込み層によるアップサンプリングと複数種類の畳み込み層による特徴抽出を繰り返し行うことで,16 kHzの特徴量を獲得する.各UpsamplingBlockにおけるパラメータを表\ref{sec4:tab:vocoder_main_params}に示す.ここで,$K$はカーネルサイズ,$S$はストライド,$R$はダイレーション,$D$は次元,$T$は系列長である.畳み込み層についてはカーネルサイズとダイレーションを集合として表記しているが,実際はこれらの直積の元,すなわち$\lr{3, 1}$や$\lr{3, 3}$,$\lr{3, 5}$をパラメータとする層が存在することを表す.すなわち,転置畳み込み層一層に対し,その後の特徴量抽出は15種類の異なるカーネルサイズ,ダイレーションを設定した畳み込み層によって行われる.最後に,畳み込み層によって1次元まで次元を削減することで,16 kHzの音声波形が得られる.
\begin{table}[bt]
    \centering
    \caption{$\vocoderMain$の各UpsamplingBlockにおけるパラメータ}
    \label{sec4:tab:vocoder_main_params}
    \begin{center}
        \renewcommand{\arraystretch}{0.9}
        \setlength{\tabcolsep}{8pt}
        \scalebox{1.0}{
            \begin{tabular}{|c|c|c|c|}
                \hline
                  & \multicolumn{1}{c|}{転置畳み込み層 $\lr{\kernelSizeUpper, \strideUpper}$} & \multicolumn{1}{c|}{畳み込み層 $\lr{\kernelSizeUpper, \dilationUpper}$} & \multicolumn{1}{c|}{出力特徴量の形状 $\lr{\dimUpper, \timeUpper}$} \\
                \hline
                1 & $\lr{11, 5}$                                                       & $\lr{\lrc{3, 5, 7, 9, 11}, \lrc{1, 3, 5}}$                         & $\lr{512, 250}$                                            \\
                2 & $\lr{8, 4}$                                                        & $\lr{\lrc{3, 5, 7, 9, 11}, \lrc{1, 3, 5}}$                         & $\lr{256, 1000}$                                           \\
                3 & $\lr{4, 2}$                                                        & $\lr{\lrc{3, 5, 7, 9, 11}, \lrc{1, 3, 5}}$                         & $\lr{128, 2000}$                                           \\
                4 & $\lr{4, 2}$                                                        & $\lr{\lrc{3, 5, 7, 9, 11}, \lrc{1, 3, 5}}$                         & $\lr{64, 4000}$                                            \\
                5 & $\lr{4, 2}$                                                        & $\lr{\lrc{3, 5, 7, 9, 11}, \lrc{1, 3, 5}}$                         & $\lr{32, 8000}$                                            \\
                6 & $\lr{4, 2}$                                                        & $\lr{\lrc{3, 5, 7, 9, 11}, \lrc{1, 3, 5}}$                         & $\lr{16, 16000}$                                           \\
                \hline
            \end{tabular}
        }
    \end{center}
\end{table}

\subsubsection{学習方法}
ネットワークAについて,最適化手法はAdamW\cite{loshchilov2017decoupled}を利用し,$\beta_{1} = 0.9, \beta_{2} = 0.98, \lambda = 0.01$とした.スケジューラはCosine Annealing with Warmupを利用し,$\learningRate_{\text{min}} = 1.0 \times 10^{-6}, \learningRate_{\text{max}} = 1.0 \times 10^{-3}, \text{warmup\_steps} = 5, \epoch_{\text{max}} = 50$とした.バッチサイズは4とし,8イテレーションに一回重みを更新するようGradient Accumulationを行った.モデルに入力する動画の秒数は10秒を上限とし,10秒を超える場合はランダムにトリミング,10秒未満の場合はゼロパディングした.また,ゼロパディングした部分は損失の計算からは除外した.勾配のノルムは3.0を上限としてクリッピングした.10エポック連続して検証データに対する損失$\lossFuncUpper_{A}$の値が小さくならない場合には学習を中断するようにし(Early Stopping),学習終了時には検証データに対する損失が最も小さかったエポックにおけるチェックポイントを保存して,これをテストデータに対する評価に用いた.

ネットワークBについて,ここではネットワークAの重みは固定した.最適化手法はAdamWを利用し,$\beta_{1} = 0.9, \beta_{2} = 0.98, \lambda = 0.01$とした.スケジューラはCosine Annealing with Warmupを利用し,$\learningRate_{\text{min}} = 1.0 \times 10^{-6}, \learningRate_{\text{max}} = 5.0 \times 10^{-4}, \text{warmup\_steps} = 5, \epoch_{\text{max}} = 50$とした.$\learningRate_{\text{max}}$をネットワークAから半減したのは,学習の安定化のためである.その他の設定はネットワークAと同様で,Early Stoppingにおいて監視したのは検証データに対する$\lossFuncUpper_{B}$の値である.

ボコーダについて,ここでははじめにHi-Fi-Captainのみを用いて学習させ,その後JVSによって再学習した.最適化手法はAdamWを利用し,$\beta_{1} = 0.8$,$\beta_{2} = 0.99$,$\lambda = \num{1.0e-5}$とした.スケジューラはExponentialLRSchedulerを利用し,$\gamma = 0.99$とした.また,最大エポック数は30とした.バッチサイズは16とした.モデルへの入力は1秒を上限とし,1秒を超える場合はランダムにトリミング,1秒未満の場合はゼロパディングした.ここではEarly Stoppingは適用せず,学習終了時に検証データに対する損失(メルスペクトログラムに対するMAE Loss)が最も小さかったエポックにおけるチェックポイントを保存し,これをテストデータに対する評価に用いた.また,先行研究\cite{choi2023intelligible}においては学習時に,メルスペクトログラムにノイズを付与するデータ拡張手法が提案されている.本研究では,動画から推定されるメルスペクトログラムとHuBERT離散特徴量の推定精度向上に焦点を当てたため,ボコーダの学習は原音声から計算される特徴量で行い,ボコーダ自体の汎化性能向上による精度改善は追求しなかった.

実装に用いた深層学習ライブラリはPyTorchおよびPyTorch Lightningである.GPUにはNVIDIA RTX A4000を利用し,計算の高速化のためAutomatic Mixed Precisionを適用した.

\subsubsection{客観評価}
合成音声の客観評価には,二種類の指標を用いた.
一つ目は,音声認識の結果から算出した単語誤り率(Word Error Rate; WER)である.WERの計算方法について,まず,正解文字列$s_{1}$と音声認識モデルによる予測文字列$s_{2}$に対し,レーベンシュタイン距離によってその差分を測る.レーベンシュタイン距離は,二つの文字列を一致させるために必要なトークンの挿入数$I$,削除数$D$,置換数$R$の和の最小値として定義される.WERは,レーベンシュタイン距離を測ることによって得られた$I, D, R$を利用し,
\begin{equation}
    \text{WER}\lr{s_{1}, s_{2}} = \frac{I + D + R}{|s_{1}|}
\end{equation}
で与えられる.ここで,$|s_{1}|$は正解文字列$s_{1}$のトークン数を表す.実際には,音声認識モデルにWhisper\cite{radford2023robust}を利用し,出力される漢字仮名交じり文に対してMeCabを用いて分かち書きを行った上で,jiwerというライブラリを用いて算出した.WhisperはLargeモデルを利用し,MeCabの辞書にはunidicを利用した.WERの値は0\%以上であり,この値が低いほど音声認識の誤りが少ないため,より聞き取りやすい音声であると判断した.

二つ目は,話者ベクトルから計算したコサイン類似度である.モデルへの入力値を計算するのに用いた話者識別モデルを同様に利用して,各サンプルごとに合成音声の話者ベクトルと原音声の話者ベクトルを計算し,これらのコサイン類似度を計算した.今回構築するモデルは4人の話者に対応するモデルとなるため,原音声に似た声質の合成音声が得られているかをこの指標で評価した.値は-1以上1以下であり,高いほど原音声と似た合成音声だと判断した.

\subsubsection{主観評価}
\label{sec4:sec:sbj_explanation}
合成音声の主観評価では,音声の明瞭性と類似性の二点を評価した.今回はクラウドワークスというクラウドソーシングサービスおよび,自作の実験用Webサイトを利用してオンラインで実験を実施した.被験者の条件は,日本語母語話者であること,聴覚に異常がないこと,イヤホンあるいはヘッドホンを用いて静かな環境で実験を実施可能であることとした.被験者の方に行っていただいた項目は,以下の五つである.
\begin{enumerate}
    \item アンケート
    \item 練習試行(明瞭性)
    \item 本番試行(明瞭性)
    \item 練習試行(類似性)
    \item 本番試行(類似性)
\end{enumerate}

一つ目のアンケートでは,被験者についての基本的な統計を取ることを目的として,性別・年齢・実験に利用した音響機器について回答してもらった.性別は,男性,女性,無回答の三つからの選択式とした.年齢は被験者の方に直接数値を入力してもらう形式とした.実験に使用した音響機器は,イヤホン,ヘッドホンの二つからの選択式とした.

二つ目の練習試行(明瞭性)および三つ目の本番試行(明瞭性)では,音声の明瞭性の評価を実施した.初めに練習試行を行っていただくことで実験内容を把握してもらい,その後本番施行を行っていただく流れとした.ここで,練習施行は何度でも実施可能とし,本番試行は一回のみ実施可能とした.
評価項目について,明瞭性は「話者の意図した発話内容を,一回の発話でどの程度聞き取ることができたか」を評価するものとした.実際の評価プロセスは以下の二段階で構成した.
\begin{enumerate}
    \item 音声サンプルのみを一回再生し,発話内容を聞き取ってもらう.
    \item 本来の発話内容を確認してもらい,聴取者が想定していた発話内容と本来の発話内容を照らし合わせ,音声の聞き取りやすさを5段階評価してもらう.
\end{enumerate}
5段階評価の回答項目は以下のようにした.
\begin{enumerate}
    \item 全く聞き取れなかった
    \item ほとんど聞き取れなかった
    \item ある程度聞き取れた
    \item ほとんど聞き取れた
    \item 完全に聞き取れた
\end{enumerate}

実験に利用した音声サンプルについて,練習試行では検証データ,本番試行ではテストデータを用いた.被験者ごとの評価サンプルの割り当て方法をアルゴリズム\ref{sec4:algo:sample-assignment}に示す.$\text{sentences}$は文章のリスト,$\text{method\_names}$が手法名のリスト,$\text{speaker\_names}$が話者名のリスト,$\text{num\_total\_respondents}$が被験者総数である.各被験者について,まず$\text{sentences}$と$\text{method\_names}$をランダムにシャッフルし,それからランダムサンプリングした$\text{speaker\_name}$を合わせて,一つのサンプルを決定するようになっている.この選択方法では,二つのことに注意した.一つ目は,各被験者がユニークな53文章を評価することである.評価の際に本来の発話内容を知ることになるため,それを知った上で同じ発話内容のサンプルが出現した場合,音声自体の明瞭性に関わらず発話内容がわかってしまう可能性があると考え,これを避けるようにした.二つ目は,各手法がなるべく均等な回数出現することである.今回は手法の比較が実験の目的となるため,各被験者がすべての手法を評価することが望ましいと判断した.また,評価に際し音声サンプルを一回だけ聞いてもらうようにした理由は,代用音声をコミュニケーションツールとして利用する場面を想定したとき,会話において何度も聞き返されることはストレスになると考えられるため,一回の発話で意図した発話内容をどの程度聞き取ってもらえるかをその手法の聞き取りやすさとして評価するべきだと考えたからである.
\begin{algorithm}
    \caption{Sample Assignment Algorithm}
    \label{sec4:algo:sample-assignment}
    \begin{algorithmic}[1]
        \Require \text{sentences}: List of sentences
        \Require \text{method\_names}: List of method names
        \Require \text{speaker\_names}: List of speaker names
        \Require \text{num\_total\_respondents}: Total number of respondents
        \State Initialize \text{all\_assignments} $\gets$ []
        \For{\text{respondent\_id} $\gets 1$ to \text{num\_total\_respondents}}
        \State Initialize \text{assignments} $\gets$ []
        \State Randomly shuffle \text{sentences}
        \State Randomly shuffle \text{method\_names}
        \For{$i \gets 0$ to $\text{len}\lr{\text{sentences}} - 1$}
        \State \text{sentence} $\gets \text{sentences}[i]$
        \State \text{method\_name} $\gets \text{method\_names}[i \bmod \text{len}\lr{\text{method\_names}}]$
        \State \text{speaker\_name} $\gets$ Randomly select from \text{speaker\_names}
        \State Append $\lr{\text{respondent\_id}, \text{sentence}, \text{method\_name}, \text{speaker\_name}}$ to \text{assignments}
        \EndFor
        \State \text{all\_assignments} $\gets \text{all\_assignments} \cup \text{assignments}$
        \EndFor
        \State \Return \text{all\_assignments}
    \end{algorithmic}
\end{algorithm}

四つ目の練習試行(類似性)および五つ目の本番試行(類似性)では,評価対象の音声と同一話者の原音声の類似性の評価を実施した.ここでも初めに練習試行を行っていただくことで実験内容を把握してもらい,その後本番施行を行っていただく流れとした.ここで,練習施行は何度でも実施可能とし,本番試行は一回のみ実施可能とした.
評価項目について,類似性は「評価対象の音声が同一話者の原音声とどれくらい似ているか」を評価するものとした.実際の評価プロセスは以下の二段階で構成した.
\begin{enumerate}
    \item 評価対象の音声と原音声を聞き比べてもらう.
    \item 評価対象の音声が原音声にどれくらい似ていたかを五段階評価してもらう.
\end{enumerate}
5段階評価の回答項目は以下のようにした.
\begin{enumerate}
    \item 全く似ていなかった
    \item あまり似ていなかった
    \item やや似ていた
    \item かなり似ていた
    \item 同じ話者に聞こえた
\end{enumerate}
実験に利用した音声サンプルおよび,被験者ごとの評価サンプルの割り当て方法は明瞭性の評価実験と同様にした.ただし,類似性評価においては,評価対象となる音声に対して同一話者の原音声をランダムに選択し,評価対象となる音声とペアで提示できるようにした.また,明瞭性評価では音声サンプルを一回だけ聞いて評価してもらうようにしたが,類似性評価では何度でも聞けるようにした.聞き取りやすさのようにコミュニケーションを想定した評価というより,単に原音声とどの程度似ているかを評価したいと考えたからである.さらに,明瞭性評価では評価時に発話内容を提示したが,類似性評価は発話内容に依存しないため,提示しなかった.類似性評価は発話内容に依存しないと考えらえるからである.加えて,類似性評価は明瞭性評価を完了した後にしか実施できないようにした.類似性評価と明瞭性評価に用いるサンプルは同一の発話内容のパターンであり,特に類似性評価では原音声を聴取できることから,本来の発話内容を完全に把握できると予想される.その上で明瞭性評価を行った場合,音声自体の明瞭性に関わらず発話内容がわかってしまう可能性があると考えられ,望ましくない.一方,類似性評価は発話内容に依存しない評価であるため,発話内容を知っていることが評価に影響を与えないと考えられる.よって,今回は明瞭性評価の後に類似性評価を行うことにした.

また,オンラインでの評価は効率よく数多くの方に評価していただけるという点でメリットがあるが,オフラインでの評価と比較して実験環境を制御することが難しく,評価品質が低下する恐れがある.これに対して,本実験では先行研究\cite{kirkland2023stuck}を参考に,評価サンプル中にダミー音声を混入させることで対策を講じた.ダミー音声は本研究で得られた合成音声とは無関係に,gTTSというライブラリを用いて生成したサンプルである.具体例として,明瞭性評価では
\begin{quote}
    これはダミー音声です.明瞭性は「3: ある程度聞き取れた」を選択してください.
\end{quote}
のような発話内容の音声を,類似性評価では
\begin{quote}
    これはダミー音声です.類似性は「1: 全く同じ話者には聞こえなかった」を選択してください.
\end{quote}
のような発話内容の音声を提示した.この時,その音声自体の明瞭性や類似性とは無関係に,必ずこの音声によって指定された評価値を選択するよう説明を与えた.本番試行においてダミー音声で指定された評価値を誤って選んだ場合は,すべての回答を無効にする旨を被験者に伝えた.実際,実験終了後にはそのようにデータを処理した.

被験者数は75人とし,謝礼は実験一回あたり40分程度要すると予想し,650円とした.
\input{section4_3.tex}
\subsection{考察}
\subsubsection{ネットワークB(Randomized)について}
\label{sec4:sec:consideration_b_randomized}
ネットワークB(Randomized)は客観評価において,メルスペクトログラムとHuBERT離散特徴量に対するロジットを入力特徴量としたネットワークB(Randomized・Mel-HuB)を上回る性能を示した.これより,ネットワークBへの入力として,HuBERT中間特徴量はメルスペクトログラムとHuBERT離散特徴量に対するロジットを組み合わせた特徴量よりも優れていたと言える.HuBERT中間特徴量は,音声波形をHuBERTの畳み込みエンコーダによって変換した結果であった.これはHuBERTの事前学習時に行われるMasked PredictionにおいてTransformer層の入力となる特徴量であり,Transformer層ではHuBERT中間特徴量のうち観測できる区間のみを利用して,マスクされた区間における教師ラベルを予測する必要がある.これより,畳み込みエンコーダは,音声波形の系列長を圧縮する過程で,Transformer層によって行われる文脈的構造の考慮に役立つ情報を,限られた次元の中にできるだけ詰め込むように最適化されると考える.よって,ネットワークB(Randomized)におけるHuBERT Transformer層はランダム初期化されてはいたものの,HuBERT中間特徴量自体が文脈的構造の考慮に適した情報を含んでいるから,メルスペクトログラムとHuBERT離散特徴量に対するロジットを入力特徴量とする場合よりも特徴抽出が行いやすくなり,結果として高い性能に達したと考えられる.

次に,ネットワークB(Randomized・A-SingleTask)は客観評価と主観評価の両面において,ネットワークB(Randomized)を上回る性能を示した.これより,HuBERT中間特徴量を予測するネットワークAは,メルスペクトログラムとHuBERT離散特徴量を同時に予測するマルチタスク学習を行わず,HuBERT中間特徴量のみを予測する方がネットワークBにとって適していると考えられる.ここで,ネットワークB(Randomized)とネットワークB(Randomized・Mel-HuB)の結果より,HuBERT中間特徴量は,メルスペクトログラムとHuBERT離散特徴量のロジットから構成した特徴量と比較して,ネットワークBの推定精度改善につながるより良い入力特徴量であった.よって,HuBERT中間特徴量と,メルスペクトログラムおよびHuBERT離散特徴量は,異なる情報を含んだ特徴量だと考えられる.この時,メルスペクトログラムおよびHuBERT離散特徴量についての損失がHuBERT中間特徴量についての損失と干渉することが,ネットワークBへの入力として用いるHuBERT中間特徴量を予測する上で,悪影響を及ぼしたと考えられる.

次に,ネットワークE2E(Pretrained)では,ネットワークB(Randomized・A-SingleTask)における重みを初期値としてネットワーク全体を再学習させることによる改善を狙ったが,これは効果的ではなかった.前述したように,HuBERT中間特徴量自体が文脈的構造の考慮に適した情報を含んでいたと考えると,HuBERT中間特徴量を継ぎ目として二段階で学習することはネットワークの表現力を損なうものではなく,ネットワーク全体を一度に最適化するのと変わらない品質での学習が可能だったと考えられる.

最後に,本実験では損失関数の重み係数$\lossWeightHubDisc$について五種類の値によるグリッドサーチを一貫して行い,検討した手法の性能は$\lossWeightHubDisc$に大きく依存することがわかった.特に,$\lossWeightHubDisc$の値を大きくすることによってHuBERT離散特徴量に対するCross Entropy Lossの低下が促進される一方で、メルスペクトログラムのMAE Lossが十分に下がらないままEarly Stoppingが発生するケースが散見され,マルチタスク学習において両方の損失をバランスよく最小化することの難しさが明らかとなった.実数値をとる重み係数についてのグリッドサーチには限界があるから,これを行うこと自体がモデルの性能の妨げになると考えられる.これに対し,マルチタスク学習の先行研究では,損失の下がり具合に基づいて適応的に重みを調整する方法\cite{chen2018gradnorm,liu2019end}が提案されている.これらは,各タスクの損失の勾配のノルムの調整にあたる.また,マルチタスク学習においては各タスクの損失が競合するNegative Transfer\cite{crawshaw2020multi}が発生する可能性があるが,これに対して各タスクの損失の勾配の向きを調整する方法\cite{yu2020gradient}も提案されている.こういった手法により,グリッドサーチおよび静的な重み係数を脱却することで,ネットワークの性能改善および実験時間の短縮が期待できる.

\subsubsection{ネットワークB(Pretrained)について}
ネットワークB(Pretrained)は、HuBERTの事前学習済み重みを用いて学習を行ったが、WERを低下させつつ話者類似度を維持することができなかったため,ネットワークB(Randomized)に劣る結果となった.しかし,これは単に事前学習済み重みが話者性の抽出に適していなかったわけではないと考える.実際,話者識別の先行研究\cite{chen2022large}では,音声波形をHuBERTによって変換した特徴量を入力として話者識別モデルを学習させた場合,窓長25 ms,シフト幅10 msで40次元のメルスペクトログラムを入力とする場合よりも,Equal Error Rateが低いことが示されている.ここで,HuBERTは事前学習時のまま重みを固定し,単なる特徴抽出器として用いられた.話者性の含まれるメルスペクトログラムに対し,話者識別をより高い精度で行える入力特徴量を音声波形から抽出できたことになるから,HuBERTが話者性抽出に適していないわけではないと考える.また,本研究ではHuBERT Transformer層の出力に話者ベクトルを結合する構成としたため,仮に話者性が損なわれていたとしてもそれを補完できるのではないかと考える.

これに対し,ネットワークB(Pretrained)が有効性を示さなかった理由は二つ考えられる.一つ目は,損失関数の重み係数$\lossWeightHubDisc$の適切な値が,グリッドサーチした範囲に存在しなかったことである.ネットワークB(Randomized)では適切な値が含まれていたが,グリッドサーチの候補は決め打ちであるから,その探索範囲に限界があった.この課題に対しては,\ref{sec4:sec:consideration_b_randomized}節で挙げた動的な重み係数の調整が有効だと考える.二つ目は,事前学習時との入力特徴量のギャップである.本実験では,HuBERTの事前学習に合わせてHuBERT中間特徴量をTransformer層への入力とした.しかし,事前学習時に用いられるマスクベクトルが存在しなかったこと,原音声から得られる特徴量ではないことが異なっており,こういったギャップが転移学習の妨げになった可能性がある.この課題に対しては,ネットワークAから予測されたHuBERT中間特徴量の一部区間に,マスクベクトルや原音声から得られたHuBERT中間特徴量を混入させることで,事前学習時の条件に近づける方法が考えられる.具体的には,学習開始時に高い混入率に設定しておいて,学習の進行に伴って徐々に混入率を低下させていき,最終的にネットワークAから予測されたHuBERT中間特徴量のみが入力となるようにスケジューリングする方法が考えられる.


\clearpage

\section{結論}
本研究では、癌などで喉頭を摘出することによって発声が不可能となった場合における新たな代用音声として、口唇動画を入力として音声を合成する,深層学習モデルを利用したアプローチを検討した。実験では,「AVHuBERTを利用したメルスペクトログラムとHuBERT離散特徴量を予測対象とするマルチタスク学習手法」をベースラインとして採用し,この手法を上回る新たなモデルを提案することで,自然音声に迫る合成音声の実現に近づくことを目的とした.提案手法は,ネットワークA,ネットワークB,ボコーダの3つから構成されるモデルとした.ネットワークAは,AVHuBERTを中心とした構成で,動画を入力としてHuBERT中間特徴量を予測する役割を担う.ネットワークBは,HuBERT Transformerを中心とした構成で,HuBERT中間特徴量を入力としてメルスペクトログラムとHuBERT離散特徴量を予測する役割を担う.ボコーダは,先行研究に基づいたHiFi-GANベースの構成で,メルスペクトログラムとHuBERT離散特徴量を入力として音声波形に変換する役割を担う.実験では,合成音声の明瞭性および,合成音声と原音声の類似性を,客観評価と主観評価の両面から評価した.

初めに客観評価によってネットワークの比較を実施し,特に提案手法におけるより良い構成を模索した.その結果,以下の4つのことがわかった.
\begin{enumerate}
    \item ネットワークBにおけるHuBERT Transformerの単純な転移学習は有効でない.
    \item ネットワークBの入力には,HuBERT中間特徴量を用いる方が,メルスペクトログラムとHuBERT離散特徴量に対するロジットを組み合わせた特徴量を用いるよりもネットワークBの予測精度が向上する.
    \item HuBERT中間特徴量を予測するネットワークAは,メルスペクトログラムとHuBERT離散特徴量を同時に予測するマルチタスク学習を行わず,HuBERT中間特徴量のみを予測するよう学習した方がネットワークBの予測精度が向上する.
    \item ネットワークAとネットワークBの学習方法について,HuBERT中間特徴量を継ぎ目として2段階で学習する場合に対し,ネットワーク全体を1度に最適化しても予測精度は改善されない.
\end{enumerate}
次に,客観評価をもとに選択した提案手法の代表と,ベースラインとを主観評価によって比較した.その結果,以下の2つの結果が得られた.
\begin{enumerate}
    \item ベースラインに対し,提案手法が明瞭性と類似性の両面において有意に高い評価を得た.
    \item ネットワークAにおいてマルチタスク学習を行った場合,マルチタスク学習を行わない場合と比較して、類似性の評価が有意に低くなった.
\end{enumerate}

以上より,提案手法は客観評価と主観評価の両面においてベースラインを上回ったと考えられる.一方,主観評価の結果より,最も良好な結果を示した動画音声合成モデルによる合成音声でも、いまだ分析合成や原音声との差は大きいことがわかった.これより,自然音声に迫る合成音声を実現するためには,さらなる改善が必要であることが明らかとなった.

今後の課題は2つ挙げられる.1つ目は,損失関数の重み係数に対するグリッドサーチからの脱却である.客観評価の結果より,本研究で検討したモデルの多くは,重み係数の値に性能が大きく左右されることがわかった.これに対し,GradNormやDynamic Weight Averagingのように,重み係数を動的に調節し,グリッドサーチよりも柔軟なチューニングを可能にすることで,モデルのさらなる精度改善および,実験時間の短縮が期待できる.2つ目は,HuBERTの転移学習方法である.本研究では,ネットワークBにおいてHuBERT Transformerの転移学習が有効でなかったため,その後はHuBERT Transformerをランダム初期化する場合のみに検討対象を絞った.しかし,適切な損失関数の重み係数がグリッドサーチの探索範囲内に含まれていなかったことや,事前学習時とfine-tuning時の入力特徴量にギャップがあったことが課題として挙げられ,その対処によってHuBERT Transformerをランダム初期化する場合を上回る可能性もある.損失関数の重み係数については、前述した動的な調整が考えられる。学習方法については例えば,学習初期にあえて原音声から得られる特徴量を混入させて事前学習時の条件に近づけ,徐々に混入率を下げてネットワークAによる予測結果のみを入力とするようスケジューリングする方法が考えられる.

\clearpage

\section*{謝辞}
\addcontentsline{toc}{section}{謝辞}
本研究に取り組む上で熱心に指導してくださり,計算環境の整備やクラウドワークスの活用など,新しい取り組みにも前向きにご協力くださった鏑木先生に心より感謝申し上げます.

また,日頃から多くの相談に乗ってくださり,研究に対する助言をいただいた藤田さんをはじめ,ゼミや中間発表において貴重なご意見をくださった河原先生,吉永先生,そして研究室の皆様にも厚く御礼申し上げます.

\clearpage

\bibliographystyle{junsrt}
% \bibliographystyle{abbrv,unsrt}
\addcontentsline{toc}{section}{参考文献}
\bibliography{library}

\end{document}