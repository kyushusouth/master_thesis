\subsection{考察}
\subsubsection{ネットワークB(Randomized)について}
\label{sec4:sec:consideration_b_randomized}
ネットワークB(Randomized)は客観評価において,メルスペクトログラムとHuBERT離散特徴量に対するロジットを入力特徴量としたネットワークB(Randomized・Mel-HuB)を上回る性能を示した.これより,ネットワークBへの入力として,HuBERT中間特徴量はメルスペクトログラムとHuBERT離散特徴量に対するロジットを組み合わせた特徴量よりも優れていたと考えられる.HuBERT中間特徴量は,音声波形をHuBERTの畳み込みエンコーダによって変換した結果であった.これはHuBERTの事前学習時に行われるMasked PredictionにおいてTransformer層の入力となる特徴量であり,Transformer層ではHuBERT中間特徴量のうち観測できる区間のみを利用して,マスクされた区間における教師ラベルを予測する必要がある.これより,畳み込みエンコーダは,音声波形の系列長を圧縮する過程で,Transformer層によって行われる文脈的構造の考慮に役立つ情報を,限られた次元の中にできるだけ詰め込むように最適化されると考える.よって,ネットワークB(Randomized)におけるHuBERT Transformer層はランダム初期化されてはいたものの,HuBERT中間特徴量自体が文脈的構造の考慮に適した情報を含んでいるから,メルスペクトログラムとHuBERT離散特徴量に対するロジットを入力特徴量とする場合よりも特徴抽出が行いやすくなり,結果として高い性能に達したと考えられる.

次に,ネットワークB(Randomized・A-SingleTask)は客観評価と主観評価の両面において,ネットワークB(Randomized)を上回る性能を示した.これより,HuBERT中間特徴量を予測するネットワークAは,メルスペクトログラムとHuBERT離散特徴量を同時に予測するマルチタスク学習を行わず,HuBERT中間特徴量のみを予測するよう学習した方が、ネットワークBに対するより良い入力特徴量を与えられると考えられる.ここで,ネットワークB(Randomized)とネットワークB(Randomized・Mel-HuB)の結果より,HuBERT中間特徴量は,メルスペクトログラムとHuBERT離散特徴量のロジットから構成した特徴量と比較して,ネットワークBの予測精度改善につながるより良い入力特徴量であった.よって,HuBERT中間特徴量と,メルスペクトログラムおよびHuBERT離散特徴量は,異なる情報を含んだ特徴量だと考えられる.結果として、ネットワークBへの入力として用いるHuBERT中間特徴量を予測する上では、メルスペクトログラムおよびHuBERT離散特徴量についての損失の干渉が悪影響を及ぼしたと考えられる。

次に,ネットワークE2E(Pretrained)では,ネットワークB(Randomized・A-SingleTask)における重みを初期値としてネットワーク全体を再学習させることでさらなる改善を狙ったが,これは有効でなかった.よって、HuBERT中間特徴量を継ぎ目として2段階で学習することはネットワークの表現力を損なうものではなかったと考えられる。HuBERT中間特徴量は元々DNNの中間表現であるから、これを2つのネットワークの継ぎ目としても表現力を損なわなかったと考えられる。

最後に,本実験では損失関数の重み係数$\lossWeightHubDisc$について5種類の値によるグリッドサーチを一貫して行い,検討した手法の性能は$\lossWeightHubDisc$に大きく依存することがわかった.特に,$\lossWeightHubDisc$の値を大きくすることによってHuBERT離散特徴量に対するCross Entropy Lossの低下が促進される一方で、メルスペクトログラムのMAE Lossが十分に下がらないままEarly Stoppingが発生するケースが散見され,マルチタスク学習において両方の損失をバランスよく最小化することの難しさが明らかとなった.実数値をとる重み係数についてのグリッドサーチには限界があるから,これを行うこと自体がモデルの性能の妨げになると考えられる.これに対し,マルチタスク学習の先行研究では,損失の下がり具合に基づいて適応的に重みを調整する方法\cite{chen2018gradnorm,liu2019end}が提案されている.これらは,各タスクの損失の勾配のノルムの調整にあたる.また,マルチタスク学習においては各タスクの損失が競合するNegative Transfer\cite{crawshaw2020multi}が発生する可能性があるが,これに対して各タスクの損失の勾配の向きを調整する方法\cite{yu2020gradient}も提案されている.こういった手法により,グリッドサーチおよび静的な重み係数を脱却することで、ネットワークの性能改善および実験時間の短縮が期待できる.

\subsubsection{ネットワークB(Pretrained)について}
ネットワークB(Pretrained)は、HuBERTの事前学習済み重みを用いて学習を行ったが、WERを低下させつつ話者類似度を維持することができなかったため,ネットワークB(Randomized)に劣る結果となった.しかし,これは単に事前学習済み重みが話者性の抽出に適していなかったわけではないと考える.実際,話者識別の先行研究\cite{chen2022large}では,音声波形をHuBERTによって変換した特徴量を入力として話者識別モデルを学習させた場合,窓長25 ms,シフト幅10 msで計算された40次元のメルスペクトログラムを入力とする場合よりも,Equal Error Rateが低いことが示されている.ここで,HuBERTは事前学習時のまま重みを固定し,単なる特徴抽出器として用いられた.話者性の含まれるメルスペクトログラムに対し,話者識別をより高い精度で行える入力特徴量を音声波形から抽出できたことになるから,HuBERTが話者性抽出に適していないわけではないと考える.また,本研究ではHuBERT Transformer層の出力に話者ベクトルを結合する構成としたため,仮に話者性が損なわれていたとしてもそれを補完できたのではないかと考える.

これに対し,ネットワークB(Pretrained)が有効性を示さなかった理由は2つ考えられる.1つ目は,損失関数の重み係数$\lossWeightHubDisc$の適切な値が,グリッドサーチした範囲に存在しなかったことである.ネットワークB(Randomized)では適切な値が含まれていたが,グリッドサーチの候補は決め打ちであるから,その探索範囲に限界があった.この課題に対しては,\ref{sec4:sec:consideration_b_randomized}節で挙げた動的な重み係数の調整が有効だと考える.2つ目は,事前学習時との入力特徴量のギャップである.本実験では,HuBERTの事前学習に合わせてHuBERT中間特徴量をTransformer層への入力とした.しかし,事前学習時に用いられるマスクベクトルが存在しなかったこと,原音声から得られる特徴量が存在しなかったことがギャップとなり、転移学習の妨げになった可能性がある.この課題に対しては,ネットワークAから予測されたHuBERT中間特徴量の一部区間に,マスクベクトルや原音声から得られたHuBERT中間特徴量を混入させることで,事前学習時の条件に近づける方法が考えられる。具体例として、ネットワークAから予測されたHuBERT中間特徴量に、原音声から得られるHuBERT中間特徴量を混入させる方法が考えられる。原音声から得られた特徴量の混入率を$p$とする。$p$が1に近いほど原音声の割合が高くなり、ネットワークBが取り組む問題は容易になると予想される。一方で、$p$を学習の進行に伴って徐々に小さくするスケジューリングを行えば、ネットワークBが取り組む問題の難易度が徐々に高まる。このような方法はカリキュラム学習と呼ばれ、タスクの難易度を段階的に調整することで効率的な学習を可能にすることが知られている。最終的に$p = 0$として転移学習を完了すれば、初めから$p = 0$で行った本実験よりも高い精度を達成できる可能性がある。
