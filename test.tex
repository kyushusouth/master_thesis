\documentclass[12pt]{jarticle}
\usepackage[utf8]{inputenc}
\usepackage[top=30truemm, bottom=30truemm, left=20truemm, right=20truemm]{geometry}
\usepackage{amsmath}
\usepackage{amssymb}
\usepackage{mathtools}
\usepackage{siunitx}
\usepackage{bm}
\usepackage[dvipdfmx]{graphicx}
\usepackage[dvipdfmx]{color}
\usepackage[subrefformat=parens]{subcaption}
\usepackage{booktabs}
\usepackage{tabularx}
\usepackage{tocloft}
\usepackage{enumerate}
\usepackage{url}
\usepackage{multirow}
\usepackage{caption}
\usepackage{enumitem}
\usepackage{makecell}
\usepackage{array}
\usepackage{algorithm}
\usepackage{algpseudocode}
\usepackage{amsthm}

\numberwithin{equation}{section}    % 数式番号にセクション番号をつける
\numberwithin{figure}{section}      % 図番号にセクション
\numberwithin{table}{section}      % 図番号にセクション

% \renewcommand{\figurename}{Fig.}    % 図 -> Fig.
% \renewcommand{\tablename}{Table }   % 表 -> Table

\renewcommand{\baselinestretch}{1.1}
\newtheorem{theorem}{Theorem}
\newtheorem{proposition}{Proposition}
\newtheorem{lemma}{Lemma}

\newlength{\figcaptionskip}
\setlength{\figcaptionskip}{5pt} % 図のキャプション間隔
\newlength{\tabcaptionskip}
\setlength{\tabcaptionskip}{-5pt} % 表のキャプション間隔
\captionsetup[figure]{skip=\figcaptionskip}
\captionsetup[table]{skip=\tabcaptionskip}

\setlist[enumerate]{topsep=5pt, partopsep=5pt, itemsep=0pt, parsep=0pt}
\setlength{\topsep}{5pt}
\setlength{\partopsep}{5pt}

\DeclareMathOperator*{\argmin}{arg\,min}

% data
\newcommand{\dataset}{\mathcal{D}}
\newcommand{\dimLower}{d}
\newcommand{\dimUpper}{D}
\newcommand{\heightLower}{h}
\newcommand{\heightUpper}{H}
\newcommand{\indexLower}{i}
\newcommand{\indexUpper}{I}
\newcommand{\inputLower}{x}
\newcommand{\inputUpper}{X}
\newcommand{\numLower}{n}
\newcommand{\numUpper}{N}
\newcommand{\outputLower}{y}
\newcommand{\outputUpper}{Y}
\newcommand{\timeLower}{t}
\newcommand{\timeUpper}{T}
\newcommand{\widthLower}{w}
\newcommand{\widthUpper}{W}

% model
\newcommand{\biasLower}{b}
\newcommand{\classLower}{c}
\newcommand{\classUpper}{C}
\newcommand{\dilationUpper}{R}
\newcommand{\numHeadLower}{h}
\newcommand{\numHeadUpper}{H}
\newcommand{\kernelSizeLower}{k}
\newcommand{\kernelSizeUpper}{K}
\newcommand{\modelLower}{f}
\newcommand{\modelUpper}{F}
\newcommand{\strideUpper}{S}
\newcommand{\weightLower}{w}
\newcommand{\weightUpper}{W}
\newcommand{\weightAndBias}{\theta}

% other
\newcommand{\binaryMaskLower}{m}
\newcommand{\binaryMaskUpper}{M}
\newcommand{\optimEmaConst}{\beta}
\newcommand{\epochLower}{e}
\newcommand{\counter}{\tau}
\newcommand{\learningRate}{\eta}
\newcommand{\lossFuncUpper}{L}
\newcommand{\normScale}{\gamma}
\newcommand{\normShift}{\beta}
\newcommand{\probaLower}{p}
\newcommand{\regConst}{\lambda}

% math
\newcommand{\closedinterval}[2]{[#1, #2]}
\newcommand{\openinterval}[2]{(#1, #2)}
\newcommand{\leftclosedinterval}[2]{[#1, #2)}
\newcommand{\rightclosedinterval}[2]{(#1, #2]}
\newcommand{\realSet}{\mathbb{R}}
\newcommand{\naturalSet}{\mathbb{N}}

% operator
\newcommand{\elemMul}{\odot}
\newcommand{\elemSum}{\oplus}


\begin{document}
\begin{theorem}
    Bonferroni法は、FWERを$\alpha$以下にできる。
\end{theorem}
\begin{proof}
    多重比較におけるファミリーを$\mathcal{F} = \{H_{0, i}\}_{i = 1}^{N}$とする。ここで、$H_{0, i}$は帰無仮説である。ここで、
    \begin{equation}
        \mathcal{I} = \{i \mid i \in \{1, \ldots, N\}, H_{0, i}は真である\}
    \end{equation}
    とし、$n_{0} = |\mathcal{I}|$とする。また、$V_{i}$を$H_{i}$が棄却される事象、すなわち$p_{i} \le \alpha / n$が成り立つ事象とする。
    \begin{equation}
        \text{FWER} = P\left(\bigcup_{i \in \mathcal{I}} V_{i}\right) \le \sum_{i \in \mathcal{I}} P(V_{i}) \le \sum_{i \in \mathcal{I}} \frac{\alpha}{n} = n_{0} \frac{\alpha}{n} \le \alpha
    \end{equation}
    ここで、$P(V_{i}) = \alpha / n$であるが、あえて以下と書いてもいいことを使った。
\end{proof}

\begin{theorem}
    Holm-Bonferroni法は、FWERを$\alpha$以下にできる。
\end{theorem}
\begin{proof}
    多重比較におけるファミリーを$\mathcal{F} = \{H_{0, i}\}_{i = 1}^{N}$とする。ここで、$H_{0, i}$は帰無仮説である。ここで、
    \begin{equation}
        \mathcal{I} = \{i \mid i \in \{1, \ldots, N\}, H_{0, i}は真である\}
    \end{equation}
    とし、$n_{0} = |\mathcal{I}|$とする。仮説検定の結果得られた$p$値を$\{p_{i}\}_{i = 1}^{N}$として、$p$値を小さい順に並び替えた数列を$\{p_{(i)}\}_{i = 1}^{N}$、これに対応するように並び替えたファミリーを$\{H_{0, (i)}\}_{i = 1}^{N}$とする。また、$i_{0} = \argmin_{i \in \mathcal{I}} p_{i}$とすると、
    \begin{equation}
        \label{eq1}
        i_{0} \le n - n_{0} + 1 \Leftrightarrow  \frac{1}{n - i_{0} + 1} \le \frac{1}{n_{0}}
    \end{equation}
    となる。ここで、$A_{i}$を、「$p_{i} = \min_{j \in \mathcal{I}} p_{j}$かつ真の帰無仮説$H_{0, i}$が誤って棄却される」が成り立つ事象とする。この時、Holm-Bonferroni法の計算方法と式\eqref{eq1}より
    \begin{equation}
        P(A_{i}) = \frac{\alpha}{n - i + 1} \le \frac{\alpha}{n_{0}}
    \end{equation}
    となる。ここで、$\bigcup_{i \in \mathcal{I}} A_{i}$に対して$\left(\bigcup_{i \in \mathcal{I}} A_{i}\right)^{c} = \bigcap_{i \in \mathcal{I}} A_{i}^{c}$を考える。これは、任意の$i \in \mathcal{I}$に対して$A_{i}^{c}$が成り立つ事象を意味し、$A_{i}^{c}$は$A_{i}$の条件の否定を満たす事象であるから、「$p_{i} \neq \min_{j \in \mathcal{I}} p_{j}$または真の帰無仮説$H_{0, i}$が誤って棄却されない」となる。定義より、ある$i_{0} \in \mathcal{I}$が存在して、$p_{i_{0}} = \min_{j \in \mathcal{I}} p_{j}$が成り立つ。よって、$H_{0, i_{0}}$は棄却されない。Holm-Bonferroni法の計算方法より、この時任意の$i \in \mathcal{I} \setminus \{i_{0}\}$に対して$H_{0, i}$も棄却されない。従って、$\bigcap_{i \in \mathcal{I}} A_{i}^{c}$は帰無仮説が真である場合、その全てが棄却されない事象を表している。以上より、$\bigcup_{i \in \mathcal{I}} A_{i}$は少なくとも一つの真の帰無仮説が棄却される事象を意味するから、
    \begin{equation}
        \text{FWER} = P\left(\bigcup_{i \in \mathcal{I}} A_{i}\right) \le \sum_{i \in \mathcal{I}} P(A_{i}) \le \sum_{i \in \mathcal{I}} \frac{\alpha}{n_{0}} = \alpha
    \end{equation}
\end{proof}
\end{document}