\documentclass[12pt]{jarticle}
\usepackage[utf8]{inputenc}
\usepackage[top=30truemm, bottom=30truemm, left=20truemm, right=20truemm]{geometry}
\usepackage{amsmath}
\usepackage{amssymb}
\usepackage{mathtools}
\usepackage{siunitx}
\usepackage{bm}
\usepackage[dvipdfmx]{graphicx}
\usepackage[dvipdfmx]{color}
\usepackage[subrefformat=parens]{subcaption}
\usepackage{booktabs}
\usepackage{tabularx}
\usepackage{tocloft}
\usepackage{enumerate}
\usepackage{url}
\usepackage{multirow}
\usepackage{caption}
\usepackage{enumitem}
\usepackage{makecell}
\usepackage{array}
\usepackage{algorithm}
\usepackage{algpseudocode}
\usepackage{amsthm}

\numberwithin{equation}{section}    % 数式番号にセクション番号をつける
\numberwithin{figure}{section}      % 図番号にセクション
\numberwithin{table}{section}      % 図番号にセクション

% \renewcommand{\figurename}{Fig.}    % 図 -> Fig.
% \renewcommand{\tablename}{Table }   % 表 -> Table

\renewcommand{\baselinestretch}{1.1}
\newtheorem{theorem}{Theorem}
\newtheorem{proposition}{Proposition}
\newtheorem{lemma}{Lemma}

\newlength{\figcaptionskip}
\setlength{\figcaptionskip}{5pt} % 図のキャプション間隔
\newlength{\tabcaptionskip}
\setlength{\tabcaptionskip}{-5pt} % 表のキャプション間隔
\captionsetup[figure]{skip=\figcaptionskip}
\captionsetup[table]{skip=\tabcaptionskip}

\setlist[enumerate]{topsep=5pt, partopsep=5pt, itemsep=0pt, parsep=0pt}
\setlength{\topsep}{5pt}
\setlength{\partopsep}{5pt}

% data
\newcommand{\dataset}{\mathcal{D}}
\newcommand{\dimLower}{d}
\newcommand{\dimUpper}{D}
\newcommand{\heightLower}{h}
\newcommand{\heightUpper}{H}
\newcommand{\indexLower}{i}
\newcommand{\indexUpper}{I}
\newcommand{\inputLower}{x}
\newcommand{\inputUpper}{X}
\newcommand{\numLower}{n}
\newcommand{\numUpper}{N}
\newcommand{\outputLower}{y}
\newcommand{\outputUpper}{Y}
\newcommand{\timeLower}{t}
\newcommand{\timeUpper}{T}
\newcommand{\widthLower}{w}
\newcommand{\widthUpper}{W}

% model
\newcommand{\biasLower}{b}
\newcommand{\classLower}{c}
\newcommand{\classUpper}{C}
\newcommand{\dilationUpper}{R}
\newcommand{\numHeadLower}{h}
\newcommand{\numHeadUpper}{H}
\newcommand{\kernelSizeLower}{k}
\newcommand{\kernelSizeUpper}{K}
\newcommand{\modelLower}{f}
\newcommand{\modelUpper}{F}
\newcommand{\strideUpper}{S}
\newcommand{\weightLower}{w}
\newcommand{\weightUpper}{W}
\newcommand{\weightAndBias}{\theta}

% other
\newcommand{\binaryMaskLower}{m}
\newcommand{\binaryMaskUpper}{M}
\newcommand{\optimEmaConst}{\beta}
\newcommand{\epochLower}{e}
\newcommand{\counter}{\tau}
\newcommand{\learningRate}{\eta}
\newcommand{\lossFuncUpper}{L}
\newcommand{\normScale}{\gamma}
\newcommand{\normShift}{\beta}
\newcommand{\probaLower}{p}
\newcommand{\regConst}{\lambda}

% math
\newcommand{\closedinterval}[2]{[#1, #2]}
\newcommand{\openinterval}[2]{(#1, #2)}
\newcommand{\leftclosedinterval}[2]{[#1, #2)}
\newcommand{\rightclosedinterval}[2]{(#1, #2]}
\newcommand{\realSet}{\mathbb{R}}
\newcommand{\naturalSet}{\mathbb{N}}

% operator
\newcommand{\elemMul}{\odot}
\newcommand{\elemSum}{\oplus}


\begin{document}
$\mathbb{R}$で考える.関数$f$の振動量は
\begin{equation}
    o(f, x) = \lim_{\delta \to \infty} \lr{\sup_{y \in I_{x}\lr{\delta}} f(y) - \inf_{y \in I_{x}\lr{\delta}} f(y)}
\end{equation}
で定義される.ここで,$I_{x}\lr{\delta} = \lr{x - \delta, x + \delta}$である.関数$f$が$x$で連続であることと,$o(f, x) = 0$が同値であることを示せる?

\begin{equation}
    \sup_{y \in I_{x}\lr{\delta}} f(y) - \inf_{y \in I_{x}\lr{\delta}} f(y) \le \epsilon
\end{equation}
が任意の$\epsilon > 0$で成り立つ時に,なぜ
\begin{equation}
    \lim_{\delta \to 0} \lr{\sup_{y \in I_{x}\lr{\delta}} f(y) - \inf_{y \in I_{x}\lr{\delta}} f(y)} = 0
\end{equation}
になるのですか?

関数$f: \mathbb{R} \to \mathbb{R}$が$x \in \mathbb{R}$で連続であることは,
\begin{equation}
    \lim_{\delta \to 0} \sup_{\lrAbs{x - y} < \delta} \lrAbs{f(y) - f(x)} = 0
\end{equation}
となるための必要十分条件であることを示したい.

まず,十分性を示す.任意の$\epsilon > 0$をとる.$f$が$x$で連続だから,
\begin{equation}
    \exists \delta_{0} > 0, \lrAbs{x - y} < \delta_{0} \Rightarrow \lrAbs{f(x) - f(y)} < \epsilon
\end{equation}
となる.ここで,$\gamma = \delta_{0}$とする.この時,$\delta < \gamma$を満たす任意の$\delta > 0$に対し,
\begin{equation}
    \lrAbs{x - y} < \delta \Rightarrow \lrAbs{f(x) - f(y)} < \epsilon
\end{equation}
となる.よって,
\begin{equation}
    \sup_{\lrAbs{x - y} < \delta} \lrAbs{f(x) - f(y)} \le \epsilon
\end{equation}
となる.$\epsilon$未満であることを示すことができればいいのだが,ここで行き詰まっている.何か間違えているか?
\end{document}