\documentclass[12pt]{jarticle}
\usepackage[utf8]{inputenc}
\usepackage[top=30truemm, bottom=30truemm, left=20truemm, right=20truemm]{geometry}
\usepackage{amsmath}
\usepackage{amssymb}
\usepackage{mathtools}
\usepackage{siunitx}
\usepackage{bm}
\usepackage[dvipdfmx]{graphicx}
\usepackage[dvipdfmx]{color}
\usepackage[subrefformat=parens]{subcaption}
\usepackage{booktabs}
\usepackage{tabularx}
\usepackage{tocloft}
\usepackage{enumerate}
\usepackage{url}
\usepackage{multirow}
\usepackage{caption}
\usepackage{enumitem}
\usepackage{makecell}
\usepackage{array}
\usepackage{algorithm}
\usepackage{algpseudocode}
\usepackage{amsthm}

\numberwithin{equation}{section}    % 数式番号にセクション番号をつける
\numberwithin{figure}{section}      % 図番号にセクション
\numberwithin{table}{section}      % 図番号にセクション

% \renewcommand{\figurename}{Fig.}    % 図 -> Fig.
% \renewcommand{\tablename}{Table }   % 表 -> Table

\renewcommand{\baselinestretch}{1.1}
\newtheorem{theorem}{Theorem}
\newtheorem{proposition}{Proposition}
\newtheorem{lemma}{Lemma}

\newlength{\figcaptionskip}
\setlength{\figcaptionskip}{5pt} % 図のキャプション間隔
\newlength{\tabcaptionskip}
\setlength{\tabcaptionskip}{-5pt} % 表のキャプション間隔
\captionsetup[figure]{skip=\figcaptionskip}
\captionsetup[table]{skip=\tabcaptionskip}

\setlist[enumerate]{topsep=5pt, partopsep=5pt, itemsep=0pt, parsep=0pt}
\setlength{\topsep}{5pt}
\setlength{\partopsep}{5pt}


\begin{document}
$(S, d)$を距離空間とし,$((S^{\ast}, d^{\ast}), \phi), ((\tilde{S}^{\ast}, \tilde{d}^{\ast}), \tilde{\phi})$は,以下の条件を満たす距離空間とする.
\begin{enumerate}
    \item $(S^{\ast}, d^{\ast})$は完備である.
    \item 任意の$x, y \in S$に対して$d(x, y) = d^{\ast}(\phi(x), \phi(y))$.
    \item $\phi(S)$は$S^{\ast}$において密である.すなわち,$\overline{\phi(S)} = S^{\ast}$
\end{enumerate}
ここで,$x^{\ast}$を$S^{\ast}$の任意の点とすれば,$((S^{\ast}, d^{\ast}), \phi)$に関する条件3によって,
\begin{equation}
    x^{\ast} = \lim_{n \to \infty} \phi(x_{n})
\end{equation}
となる$S$の点列$(x_{n})$が存在する.また,点列$(\tilde{\phi}(x_{n}))$は$(\tilde{S}^{\ast}, \tilde{d}^{\ast})$のCauchy列となり,条件1より$(\tilde{S}^{\ast}, \tilde{d}^{\ast})$は完備だから$(\tilde{\phi}(x_{n}))$は収束列である.従って,$\tilde{S}^{\ast}$において
\begin{equation}
    \tilde{x}^{\ast} = \lim_{n \to \infty} \tilde{\phi}(x_{n})
\end{equation}
が存在する.ここで,$x^{\ast} \in \S^{\ast}$に$\tilde{x}^{\ast} \in \tilde{S}^{\ast}$を対応させる$S^{\ast}$から$\tilde{S}^{\ast}$への写像を$f$とする.

この時,$(x_{n})$の取り方によらず$\tilde{x}^{\ast}$が一意に定まることを示したい.ここで,ある$S$の点列$(x_{n}), (y_{n})$が存在して,
\begin{equation}
    \tilde{x}^{\ast} = \lim_{n \to \infty} \tilde{\phi}(x_{n}), \quad \tilde{y}^{\ast} = \lim_{n \to \infty} \tilde{\phi}(y_{n})
\end{equation}
とする.この時,任意の$\epsilon > 0$に対して,
\begin{gather}
    \exists n_{0} \in \mathbb{N}, n > n_{0} \Rightarrow \tilde{d}^{\ast}(\tilde{x}^{\ast}, \tilde{\phi}(x_{n})) < \frac{\epsilon}{3} \\
    \exists n_{1} \in \mathbb{N}, n > n_{1} \Rightarrow \tilde{d}^{\ast}(\tilde{y}^{\ast}, \tilde{\phi}(y_{n})) < \frac{\epsilon}{3}
\end{gather}
ここで,
\begin{align}
    \tilde{d}^{\ast}(\tilde{x}^{\ast}, \tilde{y}^{\ast}) & \le \tilde{d}^{\ast}(\tilde{x}^{\ast}, \tilde{\phi}(x_{n})) + \tilde{d}^{\ast}(\tilde{\phi}(x_{n}), \tilde{\phi}(y_{n})) + \tilde{d}^{\ast}(\tilde{\phi}(y_{n}), \tilde{y}^{\ast}) \\
                                                         & < \frac{\epsilon}{3} + \tilde{d}^{\ast}(\tilde{\phi}(x_{n}), \tilde{\phi}(y_{n})) + \frac{\epsilon}{3}
\end{align}
ここから先が行き詰まっています.
\end{document}